\documentclass[a4paper,10pt]{article}
\usepackage{amssymb}
\usepackage[left=0.775in, right=0.775in, top=0.6in, bottom=0.775in]{geometry}
\usepackage{enumitem, hyperref, titlesec, amsmath,color, titlesec, changepage}
\hypersetup{
    colorlinks=true,
    urlcolor=blue,
    linkcolor=blue
}
\usepackage[scaled]{helvet}

\definecolor{sectionblue}{RGB}
{2,2,100} % Deep navy

\titleformat{\section}
  {\color{sectionblue}\normalfont\Large\bfseries}
  {}{0em}{}[\titlerule]
  
\renewcommand{\familydefault}{\sfdefault}
\usepackage[T1]{fontenc}

\pagenumbering{gobble}
\setlength{\parindent}{0pt}
\setlist[itemize]{itemsep=2pt, topsep=4pt}

% Formatting Section Titles
%\titleformat{\section}{\Large\bfseries}{}{0em}{}[\titlerule]

\begin{document}
\begin{center}
    {\huge \textbf{Abhijeet Anand, PhD}} \\
    {\normalsize \vspace{1.5mm}
    \textbf{Postdoctoral Fellow – Astrophysics \& Data Science, Lawrence Berkeley National Lab, USA}} 
    \vspace{-1mm}
    \begin{center}
    \normalfont
\href{mailto:abhijeetanand2011@gmail.com}{abhijeetanand2011@gmail.com} \quad | \quad 
    +1 650-664-9558 \quad | \quad Milpitas, CA \quad | \quad 
    \href{https://www.linkedin.com/in/abhijeet-anand-iisc}{LinkedIn} \quad | \quad 
    \href{https://scholar.google.com/citations?hl=en&user=MfOuq1IAAAAJ}{Scholar} \quad | \quad 
    \href{https://github.com/abhi0395}{GitHub}
    \end{center}
\end{center}
\begin{center}
\vspace*{-1.25mm}
\textit{Astrophysicist and Data scientist with 5+ years of experience in statistical modeling, machine learning, large-scale data analysis, and scientific computing. Seeking to apply analytical skills and scientific rigor to solve complex industry and business problems.}
\end{center}
\vspace*{-6mm}
\section*{Work Experience}
\textcolor{sectionblue}{\textbf{Postdoctoral Fellow}} \hfill 
\textcolor{sectionblue}{\textit{Lawrence Berkeley National Lab, USA}} \hfill 
\textcolor{sectionblue}{\textit{Sep 2022 – Present}}
\vspace{1.5mm}
\begin{adjustwidth}{1em}{0pt}

\textbf{\textcolor{sectionblue}{\underline{Machine Learning and Statistical Modeling}}}
\begin{itemize}[leftmargin=*, itemsep=2pt]
    \item Built and deployed \textbf{PCA-based minimum $\chi^2$ fitting and multi-class classification algorithm} (\href{https://github.com/desihub/redrock}{\textit{redrock}}) for 1D data, supporting the \href{https://en.wikipedia.org/wiki/Dark_Energy_Spectroscopic_Instrument}{DESI} survey — which is building the \textbf{largest 3D map of the Universe}.
    \vspace{-0.5mm}
    \item \textbf{Processed 500 independent time-series like dataset in $\sim$20 seconds}, scaling to \textbf{60M+ objects} across the full survey using multiprocessing on clusters (\href{https://www.nersc.gov/}{NERSC}/Slurm), also increasing the model success rate upto 95\% and reducing false object classification by $\sim$30\%.
\end{itemize}
\vspace{-1pt}
\textcolor{sectionblue}{\textbf{\underline{Big Data Processing}}}
\begin{itemize}[leftmargin=*, itemsep=2pt]
    \item Designed and maintained parallelized I/O pipelines (FITS, HDF5) to handle \textbf{10+ TB of structured and unstructured data daily}, improving memory efficiency and runtime (can read $\sim$100k objects within a few minutes).
\end{itemize}
\vspace{-0.5pt}
\textcolor{sectionblue}{\textbf{\underline{Project Leadership and Mentoring}}}
\begin{itemize}[leftmargin=*, itemsep=2pt]
    \item Led \textbf{two cross-functional projects} involving 15+ team members, coordinating science, software, and deployment efforts — improving classification results on a \textbf{1M-object subset} of the full survey.
     \vspace{-0.5mm}
    \item Advised on data-driven improvements to production pipelines and presented results to key stakeholders, supporting a successful \textbf{2-year project extension}. Recognized as a \textbf{top 10\% contributor} within the collaboration.
    \item Implemented \textbf{agile and scrum principles} while mentoring two graduate research projects through weekly sessions; guided students in data analysis, debugging workflows, and scientific writing.

\end{itemize}
\end{adjustwidth}
\vspace{2mm}
\textcolor{sectionblue}{\textbf{PhD Research Fellow}} \hfill 
\textcolor{sectionblue}{\textit{Max Planck Institute for Astrophysics, Germany}} \hfill 
\textcolor{sectionblue}{\textit{Sep 2018 – Jul 2022}}

\vspace{2mm}
\begin{adjustwidth}{1em}{0pt}

\textcolor{sectionblue}{\textbf{\underline{Machine Learning and Statistical Modeling}}}
\begin{itemize}[leftmargin=*, itemsep=2pt]
    \item Developed a parallelized matched-kernel signal detection pipeline for identifying localized features in 1D time-series–like data. \textbf{Increased number of identified features by 5x, signal purity to} $\gtrsim$~95\%, and \textbf{reduced runtime from weeks to a few hours for 1M objects} using HPCs.
     \vspace{-0.5mm}
    \item Applied non-linear \textbf{regression methods to model detected features and extract parameters}, using \textit{numba} for acceleration.
     \vspace{-0.5mm}
    \item Conducted large-scale hypothesis testing on \textbf{millions of observations}, applying resampling-based \textbf{bootstrapping techniques} to estimate uncertainties and quantify statistical significance of detected patterns.
\end{itemize}

\vspace{2pt}
\textcolor{sectionblue}{\textbf{\underline{Project Leadership and Scientific Impact}}}
\begin{itemize}[leftmargin=*, itemsep=2pt]
   \item Led \textbf{two large research projects}, from pipeline design to publication, resulting in a high-purity feature catalog now used by \textbf{40+ research teams globally}. Also served as a peer reviewer for leading scientific journals.
    \vspace{-0.5mm}
    \item Collaborated with computational physicists to \textbf{integrate observational and simulated datasets}, enabling new insights into the physical origins of key detected patterns.
\end{itemize}
\end{adjustwidth}
\vspace*{-4.5mm}
\section*{Technical Skills}

\textcolor{sectionblue}{\textbf{Programming:}} 
Python (NumPy, SciPy, scikit-learn, Matplotlib), Git, LaTeX, SQL, Jupyter Notebooks

\textcolor{sectionblue}{\textbf{ML \& Statistics:}} 
Statistical modeling (hypothesis testing, Monte Carlo simulations, bootstrapping); 
Machine learning (PCA, NMF, matched-filter detection, regression, clustering)

\textcolor{sectionblue}{\textbf{Data Engineering:}} 
Data pipelines, high-performance computing (Slurm, \href{https://www.nersc.gov}{NERSC}), data wrangling, unit testing

\textcolor{sectionblue}{\textbf{Open Source Contributions:}} 
\href{https://github.com/abhi0395/qsoabsfind}{qsoabsfind} (signal detection pipeline), 
\href{https://github.com/desihub/redrock}{redrock} (spectral classification and modeling), 
\href{https://github.com/desihub/desispec}{desispec} (data reduction and preprocessing)

\textcolor{sectionblue}{\textbf{Soft Skills:}} 
Mentorship, team leadership, project planning and risk assessment, cross-functional communication


\vspace*{-2mm}

\section*{Education}

\textcolor{sectionblue}{\textbf{PhD in Astrophysics}} \hfill \textit{Max Planck Institute for Astrophysics, Garching, Germany} \hfill \textit{Sep 2018 – Jul 2022}\\
\textcolor{sectionblue}{\textbf{BS - MS in Physics}} \hfill \textit{Indian Institute of Science (IISc), Bangalore, India} \hfill \textit{Aug 2012 – Jun 2017}
\end{document}
