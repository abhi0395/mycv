\documentclass[a4paper,10pt]{article}
\usepackage{amssymb}
\usepackage[left=0.775in, right=0.775in, top=0.5in, bottom=0.775in]{geometry}
\usepackage{enumitem, hyperref, titlesec, amsmath,color, titlesec, changepage}
\hypersetup{
    colorlinks=true,
    urlcolor=blue,
    linkcolor=blue
}
\usepackage[scaled]{helvet}

\definecolor{sectionblue}{RGB}
{2,2,100} % Deep navy

\titleformat{\section}
  {\color{sectionblue}\normalfont\Large\bfseries}
  {}{0em}{}[\titlerule]

\renewcommand{\familydefault}{\sfdefault}
\usepackage[T1]{fontenc}

\pagenumbering{gobble}
\setlength{\parindent}{0pt}
\setlist[itemize]{itemsep=2pt, topsep=4pt}

% Formatting Section Titles
%\titleformat{\section}{\Large\bfseries}{}{0em}{}[\titlerule]

\begin{document}
\begin{center}
    {\huge \textbf{Abhijeet Anand, PhD}} \\
    {\normalsize \vspace{1.5mm}
    \textbf{Postdoctoral Fellow – Astrophysics \& Data Science, Lawrence Berkeley National Lab, USA}}
    \vspace{-1mm}
    \begin{center}
    \normalfont
\href{mailto:abhijeetanand2011@gmail.com}{abhijeetanand2011@gmail.com} \quad |
\quad
    +1 650-664-9558 \quad | \quad Milpitas, CA \quad |
\quad
    \href{https://www.linkedin.com/in/abhijeet-anand-iisc}{LinkedIn} \quad | \quad
    \href{https://scholar.google.com/citations?hl=en&user=MfOuq1IAAAAJ}{Scholar} \quad |
\quad
    \href{https://github.com/abhi0395}{GitHub}
    \end{center}
\end{center}
\begin{center}
\vspace*{-1.25mm}
\textit{Astrophysicist and Data Scientist (PhD) with 5+ years building production-quality, scalable ML systems for very large datasets (60M+ records; 5+ TB/day). Expert in Python/pandas and statistical modeling (prediction, classification, hypothesis testing), with a track record of reducing false positives, improving accuracy, and deploying scientific libraries. Open-source contributor with 26 publications and hands-on software engineering experience.}

\end{center}
\vspace*{-6mm}
\section*{Work Experience}
\textcolor{sectionblue}{\textbf{Postdoctoral Fellow}} \hfill
\textcolor{sectionblue}{\textit{Lawrence Berkeley National Lab, USA}} \hfill
\textcolor{sectionblue}{\textit{Sep 2022 – Present}}
\vspace{1.5mm}
\begin{adjustwidth}{1em}{0pt}

\textcolor{sectionblue}{\textbf{\underline{Data Engineering, ML and Big Data}}}
\begin{itemize}[leftmargin=*, itemsep=2pt]
    \item Built and deployed scalable \textbf{large-scale predictive modeling pipeline} (\href{https://github.com/desihub/redrock}{\textit{redrock}}, multi-class classification; PCA feature engineering) on ~60M+ records, improving accuracy by 30\% and cutting false positives by 30\%.
    \item Designed \textbf{distributed real-time I/O pipelines (FITS, HDF5) to process 5+ TB/day, $\approx$100k rows/min} of structured and unstructured data, optimized for analytics in Python/pandas and SQL. Architecture is transferable to cloud-based systems (e.g., GCP, AWS).
    \item Developed and maintained \textbf{internal/external analysis libraries (open source)}, leading features, code reviews, and model improvements, and reporting workflows. Improved software quality via automated tests (unittest), documentation, and supporting reproducible analytics at scale.
\end{itemize}
\vspace{-0.5pt}
\textcolor{sectionblue}{\textbf{\underline{Project Leadership and Scientific Impact}}}
\begin{itemize}[leftmargin=*, itemsep=2pt]
    \item Led \textbf{two cross-functional projects} with 15+ team members, coordinating software and deployment efforts to improve classification results for ongoing five-year survey phase.
    \item Proposed and implemented data-driven improvements to production pipelines, securing a successful \textbf{2-year project extension}. Recognized as a top 10\% contributor within the collaboration.
    \item Mentored junior analysts/researchers on experimental design, statistical validation, and production-grade code—improving analysis quality and communication across teams.
\end{itemize}
\end{adjustwidth}
\vspace{2mm}
\textcolor{sectionblue}{\textbf{PhD Research Fellow}} \hfill
\textcolor{sectionblue}{\textit{Max Planck Institute for Astrophysics, Germany}} \hfill
\textcolor{sectionblue}{\textit{Sep 2018 – Jul 2022}}

\vspace{2mm}
\begin{adjustwidth}{1em}{0pt}

\textcolor{sectionblue}{\textbf{\underline{Data Engineering, ML and Big Data}}}
\begin{itemize}[leftmargin=*, itemsep=2pt]
    \item Built parallel \textbf{signal-processing models (matched-kernel detection)} for large sequential data, raising precision ($\geq95\%$ purity) and reducing runtime from weeks to hours on ~1M+ samples
    \item Developed and accelerated \textbf{non-linear regression pipelines (with Numba)} for parameter estimation, with rigorous model testing/validation and performance analysis.
    \item Ran large-scale statistical validation (hypothesis testing, bootstrapping) and experiment design to quantify uncertainty and build models.
\end{itemize}

\vspace{2pt}
\textcolor{sectionblue}{\textbf{\underline{Project Leadership and Scientific Impact}}}
\begin{itemize}[leftmargin=*, itemsep=2pt]
   \item Led \textbf{two large research projects}, from concept to publication, resulting in high-purity data products used by \textbf{50+ research teams worldwide.}
    \item Collaborated with computational physicists \textbf{to merge observational and simulated datasets}, enabling new insights into the physical origins of detected patterns.
    \end{itemize}
\end{adjustwidth}
\vspace*{-4.5mm}

\section*{Technical Skills}
\begin{tabular}{p{3cm} p{13cm}}
\textbf{Programming} & Python (NumPy, Pandas, SciPy, scikit-learn, Matplotlib), Git, LaTeX, Jupyter, unittest \\
\textbf{ML \& Statistics} & Predictive modeling (regression/classification), PCA/feature engineering, clustering; statistics \& probability (hypothesis testing, bootstrapping, experiment design). \\
\textbf{Data Engineering} & Parallel I/O pipelines (FITS, HDF5), HPC (Slurm, NERSC), Cloud-ready workflow design, Automated schedulers (cron) \\
\textbf{Open Source} & Maintainer/contributor: \href{https://github.com/abhi0395/qsoabsfind}{qsoabsfind}, \href{https://github.com/desihub/redrock}{redrock} (25+ GitHub stars), \href{https://github.com/desihub/desispec}{desispec} (37 GitHub stars) \\
\textbf{Soft Skills} & Mentorship, Team leadership, Cross-functional collaboration, Agile workflows
\end{tabular}

\iffalse
\section*{Technical Skills}

\textcolor{sectionblue}{\textbf{Programming:}}
Python (NumPy, SciPy, scikit-learn, Matplotlib), Git, LaTeX, SQL, Jupyter Notebooks

\textcolor{sectionblue}{\textbf{ML \& Statistics:}}
Statistical modeling (hypothesis testing, Monte Carlo simulations, bootstrapping);
machine learning (PCA, NMF, matched-filter detection, regression, clustering)

\textcolor{sectionblue}{\textbf{Data Engineering:}}
Data pipelines, high-performance computing (Slurm, \href{https://www.nersc.gov}{NERSC}), data wrangling, unit testing

\textcolor{sectionblue}{\textbf{Data Visualization:}}
Keynote, Matplotlib, Google slides, Powerpoint, Tableau

\textcolor{sectionblue}{\textbf{Open Source Contributions:}}
\href{https://github.com/abhi0395/qsoabsfind}{qsoabsfind} (signal detection pipeline),
\href{https://github.com/desihub/redrock}{redrock} (spectral classification and modeling),
\href{https://github.com/desihub/desispec}{desispec} (data reduction and preprocessing)

\textcolor{sectionblue}{\textbf{Soft Skills:}}
Mentorship, team leadership, project planning and risk assessment, cross-functional communication
\fi

\vspace*{-2mm}

\section*{Education}

\textcolor{sectionblue}{\textbf{PhD in Astrophysics}} \hfill \textit{Max Planck Institute for Astrophysics, Garching, Germany} \hfill \textit{Sep 2018 – Jul 2022}\\
\textcolor{sectionblue}{\textbf{BS - MS in Physics}} \hfill \textit{Indian Institute of Science (IISc), Bangalore, India} \hfill \textit{Aug 2012 – Jun 2017}
\end{document}