\documentclass[a4paper,10pt]{article}
\usepackage{latexsym}
\usepackage{titlesec}
\usepackage{enumitem}
\usepackage{hyperref}
\usepackage{xcolor}
\usepackage{geometry}

\geometry{margin=1in}

\titleformat{\section}{\large\bfseries}{}{0em}{}

\hypersetup{
    colorlinks=true,
    linkcolor=blue,
    filecolor=magenta,
    urlcolor=blue,
    citecolor=blue,
}

\begin{document}

\begin{center}
    {\LARGE \textbf{Abhijeet Anand, PhD}} \\
    Postdoctoral Scientist, Lawrence Berkeley National Lab, CA, USA \\
    \href{mailto:AbhijeetAnand@lbl.gov}{AbhijeetAnand@lbl.gov} \\
     \href{https://abhi0395.github.io/}{Website}  | \href{https://www.linkedin.com/in/abhijeet-anand-iisc}{LinkedIn} | \href{https://scholar.google.com/citations?hl=en&user=MfOuq1IAAAAJ}{Google Scholar} | \href{https://ui.adsabs.harvard.edu/public-libraries/YPXGQEsNQg-zR9R9YBYFXw}{NASA/ADS} | \href{https://github.com/abhi0395}{GitHub}\\
    Citizenship: Indian\\
\end{center}

\vspace{0.5cm}

\section*{Objective}
Postdoctoral scientist with a PhD in Astrophysics with 6 years of experience in research and analysis.
Proficient in mathematical modeling, big data analytics, and visualization using Python. Seeking to
apply these skills in researching and developing solutions for data-driven roles in industry that make people’s lives better.

\section*{Education}
\noindent
\textbf{PhD in Astrophysics} \\
Max Planck Institute for Astrophysics, Garching, Germany \hfill \textit{Sep 2018 - Jul 2022} \\
\begin{itemize}[noitemsep, topsep=0pt]
    \item Focus: Large-scale data analysis and modeling of gas properties and kinematics in and around galaxies using machine learning and statistical methods \href{https://edoc.ub.uni-muenchen.de/30337/}{[\textit{Thesis}]}.
\end{itemize}

\noindent\\
\textbf{MS in Physics} \\
Indian Institute of Science (IISc), Bangalore, India \hfill \textit{Jul 2016 - Jul 2017} \\
\begin{itemize}[noitemsep, topsep=0pt]
    \item Focus: Radio astronomy and spectral analysis using advanced radio interferometry techniques \href{https://raw.githubusercontent.com/abhi0395/mycv/main/files/MS_thesis.pdf}{[\textit{Thesis}]}.
\end{itemize}

\noindent\\
\textbf{BSc (Research) in Physics} \\
Indian Institute of Science, Bangalore, India \hfill \textit{Aug 2012 - Jun 2016} \\
\begin{itemize}[noitemsep, topsep=0pt]
    \item Focus: Computational and observational astrophysics, with a strong foundation in data analysis
methods \href{https://raw.githubusercontent.com/abhi0395/mycv/main/files/BS_thesis.pdf}{[\textit{Thesis}]}.
\end{itemize}

\section*{Professional Experience}
\noindent
\textbf{Postdoctoral Fellowship} \\
Lawrence Berkeley National Lab, Berkeley, CA, USA \hfill \textit{Sep 2022 - Present} \\
\begin{itemize}[noitemsep, topsep=0pt]
    \item Developed and implemented physics and mathematical models to improve measurement accuracy and finding features in large datasets using Python and machine learning techniques. 
     \item Collaborated with interdisciplinary teams to deploy scientific data reduction tools that enhance the operational efficiency of astronomical surveys. Contributor to the code base of Dark Energy Survey Instrument (DESI) codebase [\href{https://github.com/desihub}{link}]. DESI is a large collaboration of more than $\sim1000$ scientists and engineers from 65 institutions worldwide, creating the largest 3D map of our Universe.
        \item Led mentoring initiatives, guiding junior researchers in science analysis projects and fostering a collaborative research environment.
\end{itemize}

\noindent \\
\textbf{International Max Planck Research School PhD Fellowship} \\
Max Planck Institute for Astrophysics, Garching, Germany \hfill \textit{Sep 2018 - Jul 2022} \\
\begin{itemize}[noitemsep, topsep=0pt]
    \item Leveraged machine learning and statistical analysis techniques to find physical features in astronomical data to answer fundamental scientific questions.
    \item Developed the \href{https://github.com/abhi0395/qsoabsfind}{qsoabsfind} tool, showcasing expertise in coding and data analysis.
\end{itemize}

\section*{Skills}
\begin{itemize}[noitemsep, topsep=0pt]
    \item \textbf{Programming:} Python, NumPy, Matplotlib, SciPy, Astropy
    \item \textbf{Tools:} Linux, MacOS, LaTeX, Jupyter Notebooks, SLURM Manager, Git, HPC
    \item \textbf{Methods:} Mathematical Modeling, Data Science and Analysis Methods, Data Visualization, Basic ML and statistical methods
    \item \textbf{Collaborative Platforms:} GitHub (Developer and Contributor to \href{https://github.com/abhi0395/qsoabsfind}{qsoabsfind}, \href{https://github.com/desihub/redrock}{redrock}, \href{https://github.com/desihub}{DESI public codebase})
     \item \textbf{Languages:} Hindi (native), English (advanced), German (elementary)
\end{itemize}


\section*{Selected Publications}
I have published several lead and co-author papers ($\sim 17$) that have been well-cited (h-index = 9) in the astrophysics community, demonstrating the impact and relevance of my research.
\begin{itemize}[noitemsep, topsep=0pt]
    \item \textbf{Anand, A.,} \textit{et al.} (2024). "Archetype-Based Redshift Estimation for the Dark Energy Spectroscopic Instrument Survey," \textit{The Astronomical Journal}. \href{https://iopscience.iop.org/article/10.3847/1538-3881/ad60c2}{[DOI]}
    \item \textbf{Anand, A.,} \textit{et al.} (2022). "Cool circumgalactic gas in galaxy clusters: connecting the DESI legacy imaging survey and SDSS DR16 Mg II absorbers," \textit{Monthly Notices of the Royal Astronomical Society}. \href{https://doi.org/10.1093/mnras/stab871}{[DOI]}
    \item \textbf{Anand, A.,} \textit{et al.} (2021). "Characterizing the abundance, properties, and kinematics of the cool circumgalactic medium of galaxies in absorption with SDSS DR16," \textit{Monthly Notices of the Royal Astronomical Society}. \href{https://doi.org/10.1093/mnras/stac928}{[DOI]}
    \item DESI Collaboration; \textbf{Anand, A.,} \textit{et al.} (2024). "The Early Data Release of the Dark Energy Spectroscopic Instrument," \textit{The Astronomical Journal}. \href{https://doi.org/10.3847/1538-3881/ad3217}{[DOI]}
\end{itemize}

\section*{Selected Conferences \& Talks}
I have delivered several invited and contributed talks ($>20$) in national and international conferences, sharing my research with the scientific community.
\begin{itemize}[noitemsep, topsep=0pt]
\item \textbf{Invited Talk:} Americal Astronomical Society Meeting, USA, Jan 2025.
\item \textbf{Invited Talk:} Dark Energy Collaboration Summer Meeting, Marseille, France, 2024.
    \item \textbf{Invited Talk:} Cosmology and Extragalactic Seminar, MPA Garching, Germany, 2024.
\end{itemize}

\section*{Honors \& Awards}
\begin{itemize}[noitemsep, topsep=0pt]
 \item \textbf{Dark Energy Postdoctoral Fellowship} (2022-2025)
    \item \textbf{International Max Planck Research School PhD Fellowship} (2018–2022)
    \item \textbf{University Grants Commission - Junior Research Fellowship, Govt. of India} (2017-2018)
     \item \textbf{DST-Govt of India Higher Education Scholarship}  (2012-2017)
\end{itemize}

\section*{Professional services and activities}
\begin{itemize}[noitemsep, topsep=0pt]
    \item Successfully organized seminars and workshops, showcasing leadership skills
transferable to industry roles, active referee of prestigious astrophysical journals, reviewing several high impact papers.
     \item Strong communicator with experience in presenting complex scientific concepts to diverse community within the astrophysics community and for general public..
    \item Interviewed by \href{https://theinterviewportal.com/2020/03/13/astrophysicist-interview-8/}{The Interview Portal} and featured on a \href{https://www.youtube.com/watch?v=WmA_PnYLeCg}{podcast} to discuss career paths in astronomy and academia in general.
    \item Working with people on initiatives to promote diversity and inclusion (member of DEI committee
of DESI collaboration), which are critical in building more friendly and diverse environment in the
corporate world.
\end{itemize}

\end{document}
