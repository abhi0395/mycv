\documentclass[a4paper,10pt]{article}
\usepackage[left=0.8in, right=0.8in, top=0.65in, bottom=0.75in]{geometry}
\usepackage{enumitem, hyperref, titlesec}
\hypersetup{
    colorlinks=true,
    urlcolor=blue,
    linkcolor=blue
}
\usepackage[scaled]{helvet}
\renewcommand{\familydefault}{\sfdefault}
\usepackage[T1]{fontenc}

\pagenumbering{gobble}
\setlength{\parindent}{0pt}
\setlist[itemize]{itemsep=2pt, topsep=4pt}

% Formatting Section Titles
\titleformat{\section}{\Large\bfseries}{}{0em}{}[\titlerule]

\begin{document}

\begin{center}
    {\huge \textbf{Abhijeet Anand, PhD}} \\
    {\normalsize \vspace{1.5mm}
    \textbf{Postdoctoral Fellow – Astrophysics \& Data Science, Lawrence Berkeley National Lab, USA}} \\
    \vspace{1.5mm}
\href{mailto:abhijeetanand2011@gmail.com}{abhijeetanand2011@gmail.com} \quad 
    \href{https://abhi0395.github.io}{Website} \quad
    \href{https://www.linkedin.com/in/abhijeet-anand-iisc}{LinkedIn} \quad
    \href{https://scholar.google.com/citations?hl=en&user=MfOuq1IAAAAJ}{Google Scholar} \quad
    \href{https://github.com/abhi0395}{GitHub}\\
    \vspace{1.5mm}
    Mob: +1 650-664-9558
     \quad 
    Current Address: Mipitas, CA 95035, USA \quad 
\end{center}
\vspace{-2mm}
\section*{Summary}
An Astrophysicist with 5+ years of experience in data science, statistical modeling, large-scale data analysis, and visualization. Expertise in designing data pipelines, implementing machine learning algorithms, and optimizing high-performance computing (HPC) workflows. Experienced in working within large international collaborations with diverse, cross-functional teams. Seeking opportunities to apply and develop analytical skills for solving industry and business problems.
\vspace*{-3mm}
\section*{Work Experience}

\textbf{Postdoctoral Fellow } \hfill \textit{Lawrence Berkeley National Lab, Berkeley, USA} \hfill \textit{Sep 2022 – Present} 
\begin{itemize}

    \item Led 2 cross-functional projects across science and analysis teams (15+ members); reduced false positives by 10–-30\% and improved model success by 1–-5\%, impacting hundreds of thousands of objects.

    \item Designed and deployed scalable data pipelines on HPC systems (\href{https://www.nersc.gov/}{NERSC}/Slurm) to process 50+ TB of structured and unstructured data; automated final mathematical model construction for real-time use.

     \item Contributed to open-source repositories (e.g., \href{https://github.com/desihub/redrock}{redrock}), a core component of large-scale projects like \href{https://en.wikipedia.org/wiki/Dark_Energy_Spectroscopic_Instrument}{DESI}, which is building the largest 3D map of the Universe.

    \item Mentored 3 graduate researchers, supporting analysis of simulated and real data using data science methods.
    
    \end{itemize}
    
\vspace{2mm}
\textbf{Max Planck PhD Fellow} \hfill \textit{Max Planck Institute for Astrophysics, Garching, Germany} \hfill \textit{Sep 2018 – Jul 2022} 
\begin{itemize}

    \item Developed a parallelized feature detection system to analyze 1M+ high-dimensional datasets, cutting manual review time from weeks to hours and improving detection rate by 5--10x.

    \item Released open-source catalogs and tools (project: \href{https://github.com/abhi0395/qsoabsfind}{qsoabsfind}) used by 30+ external research groups for scientific analysis.

    \item Applied PCA/NMF for dimensionality reduction and feature detection, improving signal clarity and method performance.

    \item Built automated data wrangling pipelines, reducing preprocessing time and manual steps by 70--80\%.
    
\end{itemize}

\vspace{-2mm}
\section*{Technical Skills}
\textbf{Programming:} Python (NumPy, SciPy, scikit-learn, Matplotlib), Git, LaTeX, SQL, Jupyter Notebooks \\
\textbf{Data Science Methods:} Statistical Modeling, Machine Learning\\
\textbf{Data Engineering:} Data Pipelines, HPC (Slurm, \href{https://www.nersc.gov}{NERSC}), Data Wrangling\\
\textbf{Software Contributions:} \href{https://github.com/abhi0395/qsoabsfind}{qsoabsfind}, \href{https://github.com/desihub/redrock}{redrock}, \href{https://github.com/desihub/desispec}{desispec}\\
\textbf{Soft Skills:} Problem-Solving, Team Collaboration, Technical writing and communication
\vspace*{-2mm}

\section*{Education}

\textbf{PhD in Astrophysics} \hfill \textit{Max Planck Institute for Astrophysics, Garching, Germany} \hfill \textit{Sep 2018 – Jul 2022}\\
\textbf{BS - MS in Physics} \hfill \textit{Indian Institute of Science (IISc), Bangalore, India} \hfill \textit{Aug 2012 – Jun 2017}
\vspace*{-2mm}

\section*{Scientific Achievements}
\begin{itemize}
    \item Published 20+ peer-reviewed papers with 1800+ citations (h-index: 14). Full list: \href{https://ui.adsabs.harvard.edu/public-libraries/YPXGQEsNQg-zR9R9YBYFXw}{NASA/ADS}.
    \item Delivered 20+ invited and contributed talks at international conferences and workshops.
    \item Awarded “Builder” status in the \href{https://en.wikipedia.org/wiki/Dark_Energy_Spectroscopic_Instrument}{DESI} collaboration (top 10\% of 1,000+ members) for high-impact technical contributions.
    \item Reviewer for leading astrophysical journals and member of DEI committee of DESI collaboration.
\end{itemize}

\end{document}
