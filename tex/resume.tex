\documentclass[a4paper,10pt]{article}
\usepackage[left=0.8in, right=0.8in, top=0.65in, bottom=0.75in]{geometry}
\usepackage{enumitem, hyperref, titlesec}
\hypersetup{
    colorlinks=true,
    urlcolor=blue,
    linkcolor=blue
}
\usepackage[scaled]{helvet}
\renewcommand{\familydefault}{\sfdefault}
\usepackage[T1]{fontenc}

\pagenumbering{gobble}
\setlength{\parindent}{0pt}
\setlist[itemize]{itemsep=2pt, topsep=4pt}

% Formatting Section Titles
\titleformat{\section}{\Large\bfseries}{}{0em}{}[\titlerule]

\begin{document}

\begin{center}
    {\huge \textbf{Abhijeet Anand, PhD}} \\
    {\normalsize \vspace{1.5mm}
    \textbf{Postdoctoral Fellow – Astrophysics \& Data Science, Lawrence Berkeley National Lab, USA}} \\
    \vspace{1.5mm}
\href{mailto:abhijeetanand2011@gmail.com}{abhijeetanand2011@gmail.com} \quad 
    \href{https://abhi0395.github.io}{Website} \quad
    \href{https://www.linkedin.com/in/abhijeet-anand-iisc}{LinkedIn} \quad
    \href{https://scholar.google.com/citations?hl=en&user=MfOuq1IAAAAJ}{Google Scholar} \quad
    \href{https://github.com/abhi0395}{GitHub}\\
    \vspace{1.5mm}
    Mob: +1 650-664-9558
     \quad 
    Current Address: Mipitas, CA 95035, USA \quad 
\end{center}

\section*{Summary}
An Astrophysicist with 5+ years of experience in data science, statistical modeling, large-scale data analysis, and visualization. Expertise in designing data pipelines, implementing machine learning algorithms, and optimizing high-performance computing (HPC) workflows. Experienced in working within large international collaborations with diverse, cross-functional teams. Seeking opportunities to apply and develop analytical skills for solving industry and business problems.
\vspace*{-3mm}
\section*{Work Experience}

\textbf{Postdoctoral Fellow } \hfill \textit{Lawrence Berkeley National Lab, Berkeley, USA} \hfill \textit{Sep 2022 – Present} 
\begin{itemize}
    \item Developed and optimized statistical models to extract insights from large-scale datasets, applying linear algebra, statistical features, and machine learning techniques (project: \href{https://github.com/desihub/redrock}{redrock}).
    \item Designed and contributed to scalable data processing pipelines ($\sim 50\, \rm TB$) on high-performance computing clusters (\href{https://www.nersc.gov}{NERSC}), improving model reliability by 30-40\%.
    %\item (\textit{Contributor}) \textbf{redrock}: Developed PCA-based feature modeling for massive datasets, implemented on CPUs/GPUs. (\href{https://github.com/desihub/redrock}{GitHub})
    \item Built and deployed data visualization tools for structured and unstructured datasets to support high-impact research goals, such as building the largest 3D map of our Universe (\href{https://en.wikipedia.org/wiki/Dark_Energy_Spectroscopic_Instrument}{DESI}).
    \item Led collaborations with interdisciplinary teams, mentoring junior researchers and contributing to DEI initiatives.
\end{itemize}
\vspace{2mm}
\textbf{Max Planck PhD Fellow} \hfill \textit{Max Planck Institute for Astrophysics, Garching, Germany} \hfill \textit{Sep 2018 – Jul 2022} 
\begin{itemize}
    \item Applied unsupervised learning (NMF, PCA) and Gaussian kernel-based methods to high-dimensional astrophysical data for feature extraction (project: \href{https://github.com/abhi0395/qsoabsfind}{qsoabsfind}).
    %\item (\textit{Developer}) \textbf{qsoabsfind}: Designed and deployed an automated feature-detection tool for large-scale spectral datasets, significantly improving analysis efficiency. (\href{https://github.com/abhi0395/qsoabsfind}{GitHub})
    \item Implemented high-performance simulations and data visualization techniques to analyze large observational datasets.
\end{itemize}

\vspace{-3mm}
\section*{Education}

\textbf{PhD in Astrophysics} \hfill \textit{Max Planck Institute for Astrophysics, Garching, Germany} \hfill \textit{Sep 2018 – Jul 2022}\\
\textbf{BS - MS in Physics} \hfill \textit{Indian Institute of Science (IISc), Bangalore, India} \hfill \textit{Aug 2012 – Jun 2017}
\vspace*{-2mm}
\section*{Technical Skills}
\textbf{Programming:} Python (NumPy, SciPy, Pandas, scikit-learn, Matplotlib), Git, LaTeX, SQL \\
\textbf{Data Science Methods:} Statistical Modeling, Machine Learning, Probabilistic Inference\\
\textbf{Data Engineering:} Data Pipelines, HPC (Slurm, \href{https://www.nersc.gov}{NERSC}), Data Wrangling\\
\textbf{Tools:} Jupyter Notebooks, HPC (Slurm)\\
\textbf{Software Contributions:} \href{https://github.com/abhi0395/qsoabsfind}{qsoabsfind}, \href{https://github.com/desihub/redrock}{redrock}, \href{https://github.com/desihub/desispec}{desispec}\\
\textbf{Soft Skills:} Problem-Solving, Team Collaboration, Technical writing and communication
\vspace*{-2mm}
\section*{Publications \& Conferences}
\begin{itemize}
    \item Published 20+ papers, with over 1600+ citations (h-index: 13). Details: \href{https://ui.adsabs.harvard.edu/public-libraries/YPXGQEsNQg-zR9R9YBYFXw}{NASA/ADS}.
    \item Delivered 20+ invited and contributed talks at international workshops and conferences.
\end{itemize}
\vspace*{-3mm}
\section*{Professional Activities}
\begin{itemize}
    \item Reviewer for leading astrophysical journals (American Astronomical Society, Royal Astronomical Society).
    \item Featured in academic career podcasts and interviews.
    \item Active member of big scientific collaboration and contribute to the diversity and inclusion initiatives within them.
\end{itemize}

% \section*{Honors \& Awards}
% \begin{itemize}
%     \item Dark Energy Independent Postdoctoral Fellowship, LBNL (2022 – 2025, USD 90k/yr)
%     \item Max Planck PhD Fellowship (2018 – 2022, €40k/yr)
%     \item UGC Junior Research Fellowship, Govt. of India (2017 – 2018, INR 350k)
%     \item INSPIRE Scholarship, Govt. of India (2012 – 2017, INR 300k)
% \end{itemize}

\end{document}

