\documentclass[a4paper,11pt]{article}
\usepackage{latexsym}
\usepackage{titlesec}
\usepackage{enumitem}
\usepackage{hyperref}
\usepackage{xcolor}
\usepackage{geometry}

\geometry{margin=0.85in}

\titleformat{\section}{\large\bfseries}{}{0em}{}

\hypersetup{
    colorlinks=true,
    linkcolor=blue,
    filecolor=magenta,
    urlcolor=blue,
    citecolor=blue,
}

\begin{document}

\begin{center}
    {\LARGE \textbf{Abhijeet Anand, PhD}} \\
    Postdoctoral Scientist, Lawrence Berkeley National Lab, CA, USA \\
    \href{mailto:AbhijeetAnand@lbl.gov}{AbhijeetAnand@lbl.gov} \\
     \href{https://abhi0395.github.io/}{Website}  | \href{https://www.linkedin.com/in/abhijeet-anand-iisc}{LinkedIn} | \href{https://scholar.google.com/citations?hl=en&user=MfOuq1IAAAAJ}{Google Scholar} | \href{https://ui.adsabs.harvard.edu/public-libraries/YPXGQEsNQg-zR9R9YBYFXw}{NASA/ADS} | \href{https://github.com/abhi0395}{GitHub}\\
    Citizenship: Indian\\
\end{center}

\vspace{0.5cm}

\section*{Objective}
Postdoctoral scientist with a PhD in Astrophysics with 6 years of experience in quantitative data research and analysis.
Solid foundation in data science, mathematical modeling, big data analytics, and visualization using Python on HPCs. Seeking to
apply these skills in researching and developing solutions for quantitative research, data science, and analysis roles in industry. Committed to helping build a more inclusive and equitable society and working with teams with similar mindsets and goals.

\section*{Professional Experience and Projects}
\noindent
\textbf{Postdoctoral Fellowship} \\
Lawrence Berkeley National Lab, Berkeley, CA, USA \hfill \textit{Sep 2022 - Present} \\
\begin{itemize}[noitemsep, topsep=0pt]
    \item Developed and implemented physical and mathematical models to improve measurement accuracy and finding features in large datasets using linear algebra, statistical, and machine learning techniques \href{https://github.com/desihub/redrock}{link}. 
    \item I am also a support scientist for the biggest astronomical dataset on the planet and helping build the largest 3D map of our Universe.
     \item Collaborated with interdisciplinary teams to deploy scientific data reduction tools. Contributor to the scientific codebases [\href{https://github.com/desihub}{link}].
        \item Led mentoring initiatives, guiding junior researchers in science analysis projects and fostering a collaborative research environment.
\end{itemize}

% \noindent \\
% \textbf{International Max Planck Research School PhD Fellowship} \\
% Max Planck Institute for Astrophysics, Garching, Germany \hfill \textit{Sep 2018 - Jul 2022} \\
% \begin{itemize}[noitemsep, topsep=0pt]
%     \item Leveraged machine learning and statistical analysis techniques to find physical features in astronomical data to answer fundamental scientific questions.
%     \item Developed the \href{https://github.com/abhi0395/qsoabsfind}{qsoabsfind} tool, showcasing expertise in coding and data analysis.
% \end{itemize}

\section*{Education}
\noindent
\textbf{PhD in Astrophysics} \\
Max Planck Institute for Astrophysics, Garching, Germany \hfill \textit{Sep 2018 - Jul 2022} \\
\begin{itemize}[noitemsep, topsep=0pt]
    \item Leveraged a combination of linear algebra and statistical techniques to find physical features in astronomical data to answer fundamental scientific questions \href{https://edoc.ub.uni-muenchen.de/30337/}{[\textit{Thesis}]}. 
    \item Developed a method based on a combination of nonnegative matrix factorization and Gaussian kernel convolution-based adaptive signal-to-noise method \href{https://github.com/abhi0395/qsoabsfind}{(qsoabsfind)} to search for features in the astronomical spectra, showcasing expertise in developing, deploying, and maintaining scientific codebase.
\end{itemize}

\noindent\\
\textbf{BS - MS in Physics} \\
Indian Institute of Science (IISc), Bangalore, India \hfill \textit{Aug 2012 - Jul 2017} \\
\begin{itemize}[noitemsep, topsep=0pt]
    \item Worked on big data optical and radio astronomy and spectral analysis using computational techniques in python \href{https://raw.githubusercontent.com/abhi0395/mycv/main/files/MS_thesis.pdf}{[\textit{MS Thesis}]}, \href{https://raw.githubusercontent.com/abhi0395/mycv/main/files/BS_thesis.pdf}{[\textit{BS Thesis}]}.
     \item Also worked in numerical complex system dynamics to understand the nature of physical and biological systems.
\end{itemize}

\section*{Skills}
\begin{itemize}[noitemsep, topsep=0pt]
    \item \textbf{Programming:} Python, NumPy, Matplotlib, SciPy, SQL, MS-Excel
    \item \textbf{Tools:} Jupyter Notebooks, slurm, Git, HPC, Powerpoint, Keynote, LaTeX
    \item \textbf{Methods:} Mathematical Modeling, Data Science and Analysis Methods, Data Visualization, Basic statistical methods
    \item \textbf{Collaborative Platforms:} GitHub (Developer and Contributor to \href{https://github.com/abhi0395/qsoabsfind}{qsoabsfind}, \href{https://github.com/desihub/redrock}{redrock}, \href{https://github.com/desihub}{DESI public codebase})
     \item \textbf{Languages:} Hindi (native), English (advanced), German (elementary)
\end{itemize}

\section*{Selected Publications}
I have published several lead and co-author papers ($= 20$) that have been well-cited ($\geq 1430$ citations, h-index = 11) in the astrophysics community, demonstrating the impact and relevance of my research.
\begin{itemize}[noitemsep, topsep=0pt]
    \item \textbf{Anand, A.,} \textit{et al.} (2024). "Archetype-Based Redshift Estimation for the Dark Energy Spectroscopic Instrument Survey," \textit{The Astronomical Journal}. \href{https://iopscience.iop.org/article/10.3847/1538-3881/ad60c2}{[DOI]}
    \item \textbf{Anand, A.,} \textit{et al.} (2022). "Cool circumgalactic gas in galaxy clusters: connecting the DESI legacy imaging survey and SDSS DR16 Mg II absorbers," \textit{Monthly Notices of the Royal Astronomical Society}. \href{https://doi.org/10.1093/mnras/stab871}{[DOI]}
    \item \textbf{Anand, A.,} \textit{et al.} (2021). "Characterizing the abundance, properties, and kinematics of the cool circumgalactic medium of galaxies in absorption with SDSS DR16," \textit{Monthly Notices of the Royal Astronomical Society}. \href{https://doi.org/10.1093/mnras/stac928}{[DOI]}
    \item DESI Collaboration; \textbf{incl. Anand, A.,} \textit{et al.} (2024). "DESI 2024 VI: Cosmological Constraints from the Measurements of Baryon Acoustic Oscillations", \textit{ArXiv}. \href{10.48550/arXiv.2404.03002}{[DOI]}
    %\item DESI Collaboration; \textbf{Anand, A.,} \textit{et al.} (2024). "The Early Data Release of the Dark Energy Spectroscopic Instrument," \textit{The Astronomical Journal}. \href{https://doi.org/10.3847/1538-3881/ad3217}{[DOI]}
\end{itemize}

\section*{Selected Conferences \& Talks}
I have delivered several invited and contributed talks ($>20$) at national and international conferences, sharing my research with the scientific community.
\begin{itemize}[noitemsep, topsep=0pt]
\item \textbf{Invited Talk:} Joint Astronomy Seminar, IISc, Bangalore, Dec 2024.
\item \textbf{Invited Talk:} Dark Energy Collaboration Summer Meeting, Marseille, France, Jul 2024.
    \item \textbf{Invited Talk:} Cosmology and Extragalactic Seminar, MPA Garching, Germany, Feb 2024.
\end{itemize}

\section*{Professional services and activities}
\begin{itemize}[noitemsep, topsep=0pt]
    \item Active referee of prestigious astrophysical journals reviewing several high-impact papers, organizer of seminars and workshops.
     \item Strong communicator with experience in presenting complex scientific concepts to a diverse community.
    \item Interviewed by \href{https://theinterviewportal.com/2020/03/13/astrophysicist-interview-8/}{The Interview Portal} and featured on a \href{https://www.youtube.com/watch?v=WmA_PnYLeCg}{podcast} to discuss career paths in astronomy and academia in general.
    \item Working with people on initiatives to promote diversity and inclusion (member of DEI committee of DESI collaboration).
\end{itemize}

\section*{Honors \& Awards}
\begin{itemize}[noitemsep, topsep=0pt]
 \item \textbf{Dark Energy Postdoctoral Fellowship} (2022-2025)
    \item \textbf{International Max Planck Research School PhD Fellowship} (2018–2022)
    \item \textbf{University Grants Commission - Junior Research Fellowship, Govt. of India} (2017-2018)
     \item \textbf{Department of Science \& Technology-Govt of India INSPIRE Scholarship}  (2012-2017)
\end{itemize}


\end{document}
