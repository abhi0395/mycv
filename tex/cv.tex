% The formatting of this CV is based on @davidwhogg's layout.

\documentclass[12pt,letterpaper]{article}
\usepackage{color}
\usepackage{fancyhdr}
\usepackage{hyperref}
\usepackage{ifthen}
\usepackage{enumitem}

% \usepackage[yyyymmdd]{datetime}
% \renewcommand{\dateseparator}{-}

% Link formatting.
\definecolor{numcolor}{rgb}{0.5,0.5,0.5}
\definecolor{linkcolor}{rgb}{0,0,0.4}
\hypersetup{%
    colorlinks=true,        % false: boxed links; true: colored links
    linkcolor=linkcolor,    % color of internal links
    citecolor=linkcolor,    % color of links to bibliography
    filecolor=linkcolor,    % color of file links
    urlcolor=linkcolor      % color of external links
}

% Text formatting.
\newcommand{\foreign}[1]{\textit{#1}}
\newcommand{\etal}{\foreign{et~al.}}
\newcommand{\project}[1]{\textsl{#1}}
\definecolor{grey}{rgb}{0.5,0.5,0.5}
\newcommand{\deemph}[1]{\textcolor{grey}{\footnotesize{#1}}}

% literature links--use doi if you can
  \newcommand{\doi}[2]{\emph{\href{http://dx.doi.org/#1}{{#2}}}}
  \newcommand{\ads}[2]{\href{http://adsabs.harvard.edu/abs/#1}{{#2}}}
  \newcommand{\isbn}[1]{{\footnotesize(\textsc{isbn:}{#1})}}
  \newcommand{\arxiv}[1]{{\href{http://arxiv.org/abs/#1}{arXiv:{#1}}}}

% Section headings.
\renewcommand\familydefault{\sfdefault}
\usepackage{titlesec}

\titleformat{\subsection}
{\normalfont\sffamily\large\bfseries}
{}{0pt}{}

\titleformat{\subsubsection}
{\normalfont\sffamily\bfseries}
{}{0pt}{}

\titlespacing{\subsection}{0pt}{2\parskip}{0pt}
\titlespacing{\subsubsection}{0pt}{\parskip}{0pt}

\newcommand{\cvheading}[1]{\addvspace{1ex}\pagebreak[2]\par\textbf{#1}\nopagebreak\vspace{-0.4em}}

% Set up the custom unordered list.
\newcounter{refpubnum}
\newcommand{\cvlist}{%
    \rightmargin=0in
    \leftmargin=0.15in
    \topsep=0ex
    \partopsep=0pt
    \itemsep=0.2ex
    \parsep=0pt
    \itemindent=-1.0\leftmargin
    \listparindent=0.0\leftmargin
    \settowidth{\labelsep}{~}
    \usecounter{refpubnum}
}

% Margins and spaces.
\raggedright
\setlength{\oddsidemargin}{0in}
\setlength{\topmargin}{0in}
\setlength{\headsep}{0.20in}
\setlength{\headheight}{0.25in}
\setlength{\textheight}{9.1in}
\addtolength{\topmargin}{-\headsep}
\addtolength{\topmargin}{-\headheight}
\setlength{\textwidth}{6.50in}
\setlength{\parindent}{0in}
\setlength{\parskip}{1ex}

% Headings and footings.
\renewcommand{\headrulewidth}{0pt}
\pagestyle{fancy}
\lhead{\deemph{Abhijeet Anand}}
\chead{\deemph{Curriculum Vitae}}
\rhead{\deemph{\thepage}}
\cfoot{\deemph{Last updated: \today}}

% Journal names.
\newcommand{\aj}{AJ}
\newcommand{\apj}{ApJ}
\newcommand{\pasp}{PASP}
\newcommand{\mnras}{MNRAS}

\newcommand{\tex}[1]{#1}

\begin{document}\thispagestyle{empty}\sloppy\sloppypar\raggedbottom

\textbf{\Large Abhijeet Anand, PhD}\\[0.5ex]
Postdoctoral Fellow, Lawrence Berkeley National Lab, CA, USA\\[0.5ex]
\textsf{\small AbhijeetAnand@lbl.gov, \href{https://abhi0395.github.io/}{https://abhi0395.github.io/}, \href{https://scholar.google.com/citations?hl=en&user=MfOuq1IAAAAJ}{Google Scholar}}\\[0.5ex]

\subsection{Research Interests}

I am an astrophysicist and data scientist with expertise in scientific software development and large-scale spectroscopic data analysis. My research focuses on observational galaxy formation and evolution, the circumgalactic and intergalactic medium, quasar absorption lines, and cosmology. I develop scalable tools and pipelines to extract physical insights from large surveys like DESI and SDSS, with the goal of understanding gas flows around galaxies, the cosmic metal cycle, and feedback mechanisms across cosmic time.

\vspace{-1.5mm}
\subsection{Professional Experience}
\begin{list}{}{\cvlist}
\item Sep 2022 - present: Postdoctoral Fellow, Lawrence Berkeley National Lab, Berkeley, CA, USA \\
\begin{itemize}
 \item \emph{Investigating properties of metal absorbers in and around galaxies with the DESI survey.}
 \item \emph{Support Scientist for DESI spectroscopic pipeline, Improved redshift estimation algorithms for DESI galaxies, reduced false positives and catastrophic failures by $\sim 30\%$.}
  \item \emph{Contributed to data analysis and scientific interpretation for several high-impact papers within the collaboration.}
  \vspace{-1mm}
\end{itemize}
\item Sep 2018 - Jul 2022: PhD Fellow, Max Planck Institute for Astrophysics, Garching, Germany
\begin{itemize}
\item \emph{Conducted extensive studies on the gas distribution in the circumgalactic and intracluster medium using data from SDSS and DESI.}
  \vspace{-1mm}
\item \emph{Developed the quasar absorber finder, \href{https://github.com/abhi0395/qsoabsfind}{qsoabsfind}, a tool used for detecting and analyzing metal absorbers in quasar spectra.}
\end{itemize}
%\item Sep 2017 - Jul 2018: UGC - JRF\footnote{University Grants Commission, Govt. of India - Junior Research Fellowship}, National Institute of Advanced Studies, Bangalore, India
\end{list}
\vspace{-1.5mm}
\subsection{Education}
\begin{list}{}{\cvlist}
  \item Sep 2018 - Jul 2022: PhD in Astrophysics, Ludwig Maximilian University - Max Planck Institute for Astrophysics (MPA), Germany.
\begin{itemize}
    \item Thesis: \href{https://edoc.ub.uni-muenchen.de/30337/}{Probing cool and warm circumgalactic gas in galaxies and clusters with large spectroscopic and imaging surveys}
      \vspace{-1mm}
    \item Advisors: \href{https://www.mpa-garching.mpg.de/galaxyformation}{Prof. Dr. Guinevere Kauffmann} \& \href{https://www.ita.uni-heidelberg.de/~dnelson/}{Dr. Dylan Nelson}
\end{itemize}

\item Jul 2016 - Jul 2017: MS, Physics, Indian Institute of Science (IISc), Bangalore, India.
\begin{itemize}
    \item Thesis: \href{https://raw.githubusercontent.com/abhi0395/mycv/main/files/MS_thesis.pdf}{A sensitive search for HI 21 cm emission from super disks in radio galaxies}
      \vspace{-1mm}
    \item Advisor: \href{http://www.physics.iisc.ernet.in/%7Enroy/}{Prof. Nirupam Roy}
  \end{itemize}
\item Aug 2012 - June 2016: BSc (Research), Physics, Indian Institute of Science, Bangalore, India.
\begin{itemize}
    \item Thesis: \href{https://raw.githubusercontent.com/abhi0395/mycv/main/files/BS_thesis.pdf}{Sources of Continuum Opacity in Hydrogen deficient stars}
      \vspace{-1mm}
    \item Advisors: \href{https://www.iiap.res.in/?q=user/29}{Prof. Gajendra Pandey}, \href{https://www.iiap.res.in/}{Indian Institute of Astrophysics (IIA)}
  \end{itemize}
\end{list}

\vspace{-1.5mm}
\subsection{Selected visits, talks and conferences}
\vspace{-1mm}
$^i$\emph{Invited Talks}, $^c$\emph{Contributed Talks}, $^s$\emph{Schools \& Workshops}
\begin{list}{}{\cvlist}
\item $^i$ May 2025:  \emph{Astro Lunch seminar}, University of Washington, Seattle, USA, May 20, 2025 (\textit{talk})
\item $^i$ Dec 2024:  \emph{Joint Astronomy Program Seminar}, IISc, Bangalore, India, Dec 11, 2024 (\textit{talk})
\item $^c$ Dec 2024:  \emph{Baryons Beyond Galactic Boundaries}, IUCAA Pune, India, Dec 2 - Dec 6, 2024 (\textit{talk})
\item $^c$ Nov 2024:  \emph{ISSAC Conference}, St. Stephen's College, New Delhi, Nov 20 - Nov 22, 2024 (\textit{talk})
\item $^i$ Feb 2024:  \emph{Extragalactic Seminar}, MPA Garching, Germany, Feb 29, 2024 (\textit{Talk}).
\item $^i$ Feb 2024:  \emph{Cosmology Seminar}, MPA Garching, Germany, Feb 27, 2024 (\textit{Talk}).
\item $^i$ Mar 2022:  \emph{\href{https://zah.uni-heidelberg.de/research-groups\#c2659}{Computational Galaxy Formation and Evolution Group}}, ZAH/ITA, University of Heidelberg, 14-16 Mar, 2022 (\textit{Visitor}).
\item $^i$ Feb 2022:  \emph{High Energy Group Seminar}, MPE Garching, 15 Feb, 2022 (\textit{Virtual}).
\item $^i$ Feb 2022:  \emph{DESI Group Seminar}, Berkeley Lab, 1 Feb, 2022 (\textit{Virtual}).
\item $^i$ Dec 2021:  \emph{\href{https://pweb.cfa.harvard.edu/calendar/event/9298}{Galaxies \& Cosmology Seminar}}, CfA Harvard, 13 Dec, 2021 (\textit{Virtual}).
\item $^i$ Sep 2021:  \textit{STARs lab}, Arizona State University, 24 Sep, 2021 (\textit{Virtual}).
\end{list}
\vspace{-1.5mm}
\subsection{Mentoring}
  \vspace{-1mm}
\begin{list}{}{\cvlist}
\item -- Joanne Tan (MPA Garching PhD) (Metal absorbers in TNG-100), 2024-present (paper in prep.)
\item -- Shivansh Tiwari (Delhi University UG) (QSO continuum modeling), 2025-present
%\item -- Corey Dodeson (UC Berkeley UG) (short-term reading project on CGM), 2023
%\item -- Dylan Green (UC Irvine PhD) (as part of DESI mentorship program), 2023
\end{list}
  \vspace{-2mm}
\subsection{Honors \& Awards}

\begin{list}{}{\cvlist}
  \item -- DESI Builder Status (2025), Early Career Travel Grant (USD 4000).
  \item -- IMPRS PhD Fellowship, 2018--2022.
  \item -- University Grants Commission - Junior Research Fellowship of Govt. of India, 2017-2018 .
 % \item -- Department of Science \& Technology, Govt. of India -  Higher Studies Scholarship, 2012-2017.
\end{list}

\subsection{Programming skills}
\begin{list}{}{\cvlist}
\item Programming Language ---\\
   -- python (numpy, matplotlib, scipy, astropy, scikit-learn), bash, SQL
  \item Tools \& Workflows ---\\
  -- Git, Jupyter, slurm (NERSC), LaTeX, Unix/Linux  
\item Open-source Contributions ---\\
   -- \href{https://github.com/abhi0395/qsoabsfind}{qsoabsfind}: QSO metal absorber finder (\textit{Developer})\\
   -- \href{https://github.com/desihub/redrock}{redrock}: Redshift fitter for DESI (\textit{Contributor}).\\
   -- \href{https://github.com/desihub/desispec}{desispec}: DESI spectroscopic pipeline(\textit{Contributor}).
\end{list}

% \subsection{Awarded Supercomputing time}
% \begin{list}{}{\cvlist}
% \item -- 2024: Perlmutter Supercomputer at NERSC, \textit{Properties of         precious metal absorbers in DESI quasar spectra}, 1000 CPU hours, 1000 GPU      hours, PI: \textbf{A. Anand}
% \item -- 2023: Perlmutter Supercomputer at NERSC,  \textit{Archetype based      redshift estimation for DESI}, 1000 CPU hours, 1000 GPU hours, PI: \textbf{A.   Anand}
% \item -- 2018-2022: Freya Supercomputer at MPCDF,  \textit{The multiphase       galactic halo of galaxies and clusters with large spectroscopic and imaging     surveys}, 5000 CPU hours, PI: \textbf{A. Anand}
% \end{list}

%\subsection{Observing and data processing experience}
%\begin{list}{}{\cvlist}
%\item -- Sep 2022: Support Scientist for DESI spectroscopic pipeline
%\item -- Aug 2023: Remote Support Observing Scientist for DESI, Aug 12-14, 2023.
%\item -- Jul 2024: Support Observing Scientist for DESI, Kitt Peak National Observatory, Jul 2-5, 2024.
%\end{list}

\vspace{-2mm}
\subsection{Professional services \& activities}

\begin{list}{}{\cvlist}
    \item Referee --- \\
    -- Astronomy \& Astrophysics, Astrophysical Journal, Internal Reviewer for DESI papers
    \item Member \& Organizer--- \\
        -- Core committee member of the DESI Professional Development Committee. \\
        -- \href{https://inpa.lbl.gov/events/}{INPA Weekly Seminar}, Physics Division, LBNL, 2023-2024 \\
      % \item Teaching Experience --- \\
        %-- Served as online physics tutor at E-acharya (>20 students)\footnote{an initiative to help poor students in suburban and rural areas of Bihar, India.} from June 2017 - Aug 2018. \\
        \end{list}
  \vspace{-2mm}
\subsection{Outreach activities}

\begin{list}{}{\cvlist}
    \item Press releases --- \\
        -- My papers (\href{https://www.mpa-garching.mpg.de/964620/hl202107}{I}, \href{https://www.mpa-garching.mpg.de/1066558/hl202211?c=27981}{II}) featured in the monthly research highlight of MPA and appeared on \href{https://astrobites.org/2021/05/06/cool-metal-gas-search-thanks-it-was-automated/}{astrobites}.
    \item Interviews \& podcast --- \\
      -- Interviewed by \href{https://www.youtube.com/watch?v=WmA_PnYLeCg}{Vidpeds podcast} to discuss my astronomy journey and how people from small cities can become successful astronomers. \\
      -- Gave \href{https://theinterviewportal.com/2020/03/13/astrophysicist-interview-8/}{interview} to \href{https://theinterviewportal.com/}{The Interview Portal} to help Indian college students plan their careers.
  \end{list}

\ifdefined\withpubs
  \subsection{Publications}
  Total: 25 / refereed: 18 / first author: 5 / citations: 2,929 / h-index: 17 (Last updated: 2025-07-29), List attached below

  \subsubsection{First-author publications}
\begin{list}{}{\cvlist}
  \item[{\color{numcolor}\scriptsize5}] \textbf{Anand, Abhijeet}; Aguilar, J.; Ahlen, S.; Bianchi, D.; Brodzeller, A.; \etal, 2025, \doi{10.3847/1538-4357/adef3c}{The Cosmic Evolution of C IV Absorbers at 1.4 < z < 4.5: Insights from 100,000 Systems in DESI Quasars}, The Astrophysical Journal, \textbf{990}, 151 (\arxiv{2504.20299}) [\href{https://ui.adsabs.harvard.edu/abs/2025ApJ...990..151A}{1 citations}]

\item[{\color{numcolor}\scriptsize4}] \textbf{Anand, Abhijeet}; Guy, Julien; Bailey, Stephen; Moustakas, John; Aguilar, J.; \etal, 2024, \doi{10.3847/1538-3881/ad60c2}{Archetype-based Redshift Estimation for the Dark Energy Spectroscopic Instrument Survey}, The Astronomical Journal, \textbf{168}, 124 (\arxiv{2405.19288}) [\href{https://ui.adsabs.harvard.edu/abs/2024AJ....168..124A}{25 citations}]

\item[{\color{numcolor}\scriptsize3}] \textbf{Anand, Abhijeet}; Kauffmann, Guinevere; \& Nelson, Dylan, 2022, \doi{10.1093/mnras/stac928}{Cool circumgalactic gas in galaxy clusters: connecting the DESI legacy imaging survey and SDSS DR16 Mg II absorbers}, Monthly Notices of the Royal Astronomical Society, \textbf{513}, 3210 (\arxiv{2201.07811}) [\href{https://ui.adsabs.harvard.edu/abs/2022MNRAS.513.3210A}{26 citations}]

\item[{\color{numcolor}\scriptsize2}] \textbf{Anand, Abhijeet}; Nelson, Dylan; \& Kauffmann, Guinevere, 2021, \doi{10.1093/mnras/stab871}{Characterizing the abundance, properties, and kinematics of the cool circumgalactic medium of galaxies in absorption with SDSS DR16}, Monthly Notices of the Royal Astronomical Society, \textbf{504}, 65 (\arxiv{2103.15842}) [\href{https://ui.adsabs.harvard.edu/abs/2021MNRAS.504...65A}{51 citations}]

\item[{\color{numcolor}\scriptsize1}] \textbf{Anand, Abhijeet}; Roy, Nirupam; \& Gopal-Krishna, 2019, \doi{10.1088/1674-4527/19/6/83}{Search for H I emission from superdisk candidates associated with radio galaxies}, Research in Astronomy and Astrophysics, \textbf{19}, 083 (\arxiv{1812.06875}) [\href{https://ui.adsabs.harvard.edu/abs/2019RAA....19...83A}{2 citations}] 
  %\newline \\
 %\input{in_prep}
\end{list}

\subsubsection{Significant contributions}
\begin{list}{}{\cvlist}
  \item[{\color{numcolor}\scriptsize6}] Ross, A. J.; Aguilar, J.; Ahlen, S.; Alam, S.; \textbf{Anand, Abhijeet}; \etal, 2025, \doi{10.1088/1475-7516/2025/01/125}{The construction of large-scale structure catalogs for the Dark Energy Spectroscopic Instrument}, Journal of Cosmology and Astroparticle Physics, \textbf{2025}, 125 (\arxiv{2405.16593}) [\href{https://ui.adsabs.harvard.edu/abs/2025JCAP...01..125R}{9 citations}]

\item[{\color{numcolor}\scriptsize5}] Chang, Yu-Ling; Lan, Ting-Wen; Prochaska, J. Xavier; Napolitano, Lucas; \textbf{Anand, Abhijeet}; \etal, 2024, \doi{10.3847/1538-4357/ad6c44}{Probing the Impact of Radio-mode Feedback on the Properties of the Cool Circumgalactic Medium}, The Astrophysical Journal, \textbf{974}, 191 (\arxiv{2405.08314})

\item[{\color{numcolor}\scriptsize4}] Galiullin, Ilkham; Rodriguez, Antonio C.; El-Badry, Kareem; Szkody, Paula; \textbf{Anand, Abhijeet}; \etal, 2024, \doi{10.1051/0004-6361/202450734}{Searching for new cataclysmic variables in the Chandra Source Catalog}, Astronomy and Astrophysics, \textbf{690} (\arxiv{2408.00078}) [\href{https://ui.adsabs.harvard.edu/abs/2024A&A...690A.374G}{1 citations}]

\item[{\color{numcolor}\scriptsize3}] Wu, X.; Cai, Z.; Lan, T. -W.; Zou, S.; \textbf{Anand, Abhijeet}; \etal, 2024, \doi{10.48550/arXiv.2407.17809}{Tracing the evolution of the cool gas in CGM and IGM environments through Mg II absorption from redshift z=0.75 to z=1.65 using DESI-Y1 data}, ArXiv (\arxiv{2407.17809}) [\href{https://ui.adsabs.harvard.edu/abs/2024arXiv240717809W}{3 citations}]

\item[{\color{numcolor}\scriptsize2}] Napolitano, Lucas; Pandey, Agnesh; Myers, Adam D.; Lan, Ting-Wen; \textbf{Anand, Abhijeet}; \etal, 2023, \doi{10.3847/1538-3881/ace62c}{Detecting and Characterizing Mg II Absorption in DESI Survey Validation Quasar Spectra}, The Astronomical Journal, \textbf{166}, 99 (\arxiv{2305.20016}) [\href{https://ui.adsabs.harvard.edu/abs/2023AJ....166...99N}{16 citations}]

\item[{\color{numcolor}\scriptsize1}] Ayromlou, Mohammadreza; Kauffmann, Guinevere; \textbf{Anand, Abhijeet}; \& White, Simon D. M., 2023, \doi{10.1093/mnras/stac3637}{The physical origin of galactic conformity: from theory to observation}, Monthly Notices of the Royal Astronomical Society, \textbf{519}, 1913 (\arxiv{2207.02218}) [\href{https://ui.adsabs.harvard.edu/abs/2023MNRAS.519.1913A}{16 citations}]
\end{list}

\subsubsection{Collaboration Papers}
\begin{list}{}{\cvlist}
  \item[{\color{numcolor}\scriptsize8}] Scholte, Dirk; Saintonge, Am{\'e}lie; Moustakas, John; Catinella, Barbara; Zou, Hu; \etal\ (incl.\ \textbf{A. Anand}), 2024, \doi{10.1093/mnras/stae2477}{The atomic gas sequence and mass-metallicity relation from dwarfs to massive galaxies}, Monthly Notices of the Royal Astronomical Society (\arxiv{2408.03996}) [\href{https://ui.adsabs.harvard.edu/abs/2024MNRAS.tmp.2425S}{1 citations}]

\item[{\color{numcolor}\scriptsize7}] DESI Collaboration; Adame, A. G.; Aguilar, J.; Ahlen, S.; Alam, S.; \etal\ (incl.\ \textbf{A. Anand}), 2024, \doi{10.3847/1538-3881/ad3217}{The Early Data Release of the Dark Energy Spectroscopic Instrument}, The Astronomical Journal, \textbf{168}, 58 (\arxiv{2306.06308}) [\href{https://ui.adsabs.harvard.edu/abs/2024AJ....168...58D}{207 citations}]

\item[{\color{numcolor}\scriptsize6}] Ross, A. J.; Aguilar, J.; Ahlen, S.; Alam, S.; \textbf{Anand, Abhijeet}; \etal, 2024, \doi{10.48550/arXiv.2405.16593}{The Construction of Large-scale Structure Catalogs for the Dark Energy Spectroscopic Instrument}, ArXiv (\arxiv{2405.16593}) [\href{https://ui.adsabs.harvard.edu/abs/2024arXiv240516593R}{4 citations}]

\item[{\color{numcolor}\scriptsize5}] DESI Collaboration; Adame, A. G.; Aguilar, J.; Ahlen, S.; Alam, S.; \etal\ (incl.\ \textbf{A. Anand}), 2024, \doi{10.48550/arXiv.2404.03002}{DESI 2024 VI: Cosmological Constraints from the Measurements of Baryon Acoustic Oscillations}, ArXiv (\arxiv{2404.03002}) [\href{https://ui.adsabs.harvard.edu/abs/2024arXiv240403002D}{402 citations}]

\item[{\color{numcolor}\scriptsize4}] DESI Collaboration; Adame, A. G.; Aguilar, J.; Ahlen, S.; Alam, S.; \etal\ (incl.\ \textbf{A. Anand}), 2024, \doi{10.48550/arXiv.2404.03000}{DESI 2024 III: Baryon Acoustic Oscillations from Galaxies and Quasars}, ArXiv (\arxiv{2404.03000}) [\href{https://ui.adsabs.harvard.edu/abs/2024arXiv240403000D}{132 citations}]

\item[{\color{numcolor}\scriptsize3}] DESI Collaboration; Adame, A. G.; Aguilar, J.; Ahlen, S.; Alam, S.; \etal\ (incl.\ \textbf{A. Anand}), 2024, \doi{10.48550/arXiv.2404.03001}{DESI 2024 IV: Baryon Acoustic Oscillations from the Lyman Alpha Forest}, ArXiv (\arxiv{2404.03001}) [\href{https://ui.adsabs.harvard.edu/abs/2024arXiv240403001D}{115 citations}]

\item[{\color{numcolor}\scriptsize2}] DESI Collaboration; Adame, A. G.; Aguilar, J.; Ahlen, S.; Alam, S.; \etal\ (incl.\ \textbf{A. Anand}), 2024, \doi{10.3847/1538-3881/ad0b08}{Validation of the Scientific Program for the Dark Energy Spectroscopic Instrument}, The Astronomical Journal, \textbf{167}, 62 (\arxiv{2306.06307}) [\href{https://ui.adsabs.harvard.edu/abs/2024AJ....167...62D}{141 citations}]

\item[{\color{numcolor}\scriptsize1}] Han, Jiwon Jesse; Dey, Arjun; Price-Whelan, Adrian M.; Najita, Joan; Schlafly, Edward F.; \etal\ (incl.\ \textbf{A. Anand}), 2023, \doi{10.48550/arXiv.2306.11784}{NANCY: Next-generation All-sky Near-infrared Community surveY}, ArXiv (\arxiv{2306.11784}) [\href{https://ui.adsabs.harvard.edu/abs/2023arXiv230611784H}{3 citations}]
\end{list}
\fi
\end{document}
