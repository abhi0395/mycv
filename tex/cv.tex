% The formatting of this CV is based on @davidwhogg's layout.

\documentclass[12pt,letterpaper]{article}
\input{cvstyle}
\newcommand{\tex}[1]{#1}

\begin{document}\thispagestyle{empty}\sloppy\sloppypar\raggedbottom

\textbf{\Large Abhijeet Anand, PhD}\\[0.5ex]
Institute Postdoctoral Fellow, IUCAA, Pune, India\\[0.5ex]
\textsf{\small abhijeet.anand@iucaa.in, \href{https://abhi0395.github.io/}{https://abhi0395.github.io/}, \href{https://scholar.google.com/citations?hl=en&user=MfOuq1IAAAAJ}{Google Scholar}}\\[0.5ex]

\subsection{Research Interests}

Astrophysicist specializing in galaxy formation, the circumgalactic/intergalactic medium, and quasar absorption lines. My work combines large-scale surveys (DESI, SDSS) with data science and computational methods to understand cosmic gas flows and the metal cycle in the Universe at different epochs. My long-term vision is to build an independent research program at the intersection of large surveys, data-intensive pipelines, and galaxy–CGM/IGM connections.

\vspace{-1.5mm}
\subsection{Professional Experience}
\begin{list}{}{\cvlist}
\item Dec 2025 - present: Institute Postdoctoral Fellow, IUCAA, Pune, India \\
\begin{itemize}
\item \emph{Leading/Led independent projects on physical properties of quasar absorbers in and around galaxies using MUSE, SDSS and DESI.}
 \item \emph{Builder and support scientist for the DESI spectroscopic pipeline.}
  \vspace{-1mm}
\end{itemize}
\item Sep 2022 - Oct 2025: Postdoctoral Fellow, Lawrence Berkeley National Lab, Berkeley, CA, USA \\
\begin{itemize}
\item \emph{Leading/Led independent projects on metal absorbers in and around galaxies using DESI, producing the largest metal absorber catalogs.}
 \item \emph{Developer and support scientist for the DESI spectroscopic pipeline; improved galaxy redshift estimation algorithms, reducing catastrophic failures by $\sim30\%$.}
  \item \emph{Contributed to high-impact DESI collaboration papers, while also initiating independent science projects within and outside the collaboration.}
  \vspace{-1mm}
\end{itemize}
\item Sep 2018 - Jul 2022: PhD Fellow, Max Planck Institute for Astrophysics, Garching, Germany
\begin{itemize}
\item \emph{Conducted extensive studies on the gas distribution in the circumgalactic and intracluster medium using data from SDSS and DESI.}
  \vspace{-1mm}
\item \emph{Developed the quasar absorber finder, \href{https://github.com/abhi0395/qsoabsfind}{qsoabsfind}, a tool used for detecting and analyzing metal absorbers in quasar spectra.}
\end{itemize}
\item Sep 2017 - Jul 2018: UGC - JRF\footnote{University Grants Commission, Govt. of India - Junior Research Fellowship}, National Institute of Advanced Studies, Bangalore, India
\end{list}
\vspace{-1.5mm}
\subsection{Education}
\begin{list}{}{\cvlist}
  \item Sep 2018 - Jul 2022: PhD in Astrophysics, Ludwig Maximilian University - Max Planck Institute for Astrophysics (MPA), Germany.
\begin{itemize}
    \item Thesis: \href{https://edoc.ub.uni-muenchen.de/30337/}{Probing cool and warm circumgalactic gas in galaxies and clusters with large spectroscopic and imaging surveys}
      \vspace{-1mm}
    \item Advisors: \href{https://www.mpa-garching.mpg.de/galaxyformation}{Prof. Dr. Guinevere Kauffmann} \& \href{https://www.ita.uni-heidelberg.de/~dnelson/}{Dr. Dylan Nelson}
\end{itemize}

\item Jul 2016 - Jul 2017: MS, Physics, Indian Institute of Science (IISc), Bangalore, India.
\begin{itemize}
    \item Thesis: \href{https://raw.githubusercontent.com/abhi0395/mycv/main/files/MS_thesis.pdf}{A sensitive search for HI 21 cm emission from super disks in radio galaxies}
      \vspace{-1mm}
    \item Advisor: \href{http://www.physics.iisc.ernet.in/%7Enroy/}{Prof. Nirupam Roy}
  \end{itemize}
\item Aug 2012 - June 2016: BSc (Research), Physics, Indian Institute of Science, Bangalore, India.
\begin{itemize}
    \item Thesis: \href{https://raw.githubusercontent.com/abhi0395/mycv/main/files/BS_thesis.pdf}{Sources of Continuum Opacity in Hydrogen deficient stars}
      \vspace{-1mm}
    \item Advisors: \href{https://www.iiap.res.in/?q=user/29}{Prof. Gajendra Pandey}, \href{https://www.iiap.res.in/}{Indian Institute of Astrophysics (IIA)}
  \end{itemize}
\end{list}

\vspace{-1.5mm}
\subsection{Selected visits, talks and conferences}
\vspace{-1mm}
$^i$\emph{Invited Talks}, $^c$\emph{Contributed Talks}, $^s$\emph{Schools \& Workshops}
\begin{list}{}{\cvlist}
\item $^i$ June 2025:  \emph{Astro Webinar}, Raman Research Institute, Bangalore, India, June 23, 2025 (\textit{Virtual})
\item $^i$ May 2025:  \emph{Astro Lunch seminar}, University of Washington, Seattle, USA, May 20, 2025 (\textit{talk})
\item $^i$ Dec 2024:  \emph{Joint Astronomy Program Seminar}, IISc, Bangalore, India, Dec 11, 2024 (\textit{talk})
\item $^c$ Dec 2024:  \emph{Baryons Beyond Galactic Boundaries}, IUCAA Pune, India, Dec 2 - Dec 6, 2024 (\textit{talk})
\item $^c$ Nov 2024:  \emph{ISSAC Conference}, St. Stephen's College, New Delhi, Nov 20 - Nov 22, 2024 (\textit{talk})
\item $^i$ Feb 2024:  \emph{Extragalactic Seminar}, MPA Garching, Germany, Feb 29, 2024 (\textit{talk}).
\item $^i$ Feb 2024:  \emph{Cosmology Seminar}, MPA Garching, Germany, Feb 27, 2024 (\textit{talk}).
\item $^i$ Mar 2022:  \emph{\href{https://zah.uni-heidelberg.de/research-groups\#c2659}{Computational Galaxy Formation and Evolution Group}}, ZAH/ITA, University of Heidelberg, 14-16 Mar 2022 (\textit{Visitor}).
\item $^i$ Feb 2022:  \emph{High Energy Group Seminar}, MPE Garching, 15 Feb, 2022 (\textit{Virtual}).
\item $^i$ Feb 2022:  \emph{DESI Group Seminar}, Berkeley Lab, 1 Feb, 2022 (\textit{Virtual}).
\item $^i$ Dec 2021:  \emph{\href{https://pweb.cfa.harvard.edu/calendar/event/9298}{Galaxies \& Cosmology Seminar}}, CfA Harvard, 13 Dec, 2021 (\textit{Virtual}).
\item $^i$ Sep 2021:  \textit{STARs lab}, Arizona State University, 24 Sep, 2021 (\textit{Virtual}).
\end{list}
\vspace{-1.5mm}
\subsection{Mentoring}
  \vspace{-1mm}
\begin{list}{}{\cvlist}
\item -- Mentored Joanne Tan (MPA PhD): project on metal absorbers in TNG-100 simulations (manuscript in prep. for MNRAS).
\item -- Supervised Shivansh Tiwari (UG, Delhi University): developed QSO continuum models for DESI data analysis, 2025-present.
%\item -- Corey Dodeson (UC Berkeley UG) (short-term reading project on CGM), 2023
%\item -- Dylan Green (UC Irvine PhD) (as part of DESI mentorship program), 2023
\end{list}
  \vspace{-2mm}
\subsection{Honors \& Awards}

\begin{list}{}{\cvlist}
\item -- Core member of the DESI Data Team; the DESI Collaboration awarded the 2026 \href{https://aas.org/press/desi-collaboration-receive-2026-berkeley-prize}{Berkeley Prize} of American Astronomical Society.
  \item -- \href{https://www.desi.lbl.gov/collaboration/desi-builders/}{DESI Builder Award (2025)} \textit{for outstanding contributions to the data systems operations.}
  \item --  DESI Early Career Travel Grant (USD 5000).
  \item -- \href{https://www.imprs-astro.mpg.de/content/student-class-2018-2021.html}{IMPRS PhD Fellowship}, 2018--2022.
  \item -- University Grants Commission - Junior Research Fellowship of Govt. of India, 2017-2018.
  \item -- Department of Science \& Technology, Govt. of India -  Higher Studies Scholarship, 2012-2017.
\end{list}

\subsection{Programming skills}
\begin{list}{}{\cvlist}
\item Programming Language ---\\
   -- Python (numpy, scipy, astropy, scikit-learn, matplotlib), bash, SQL — with focus on high-performance, large-scale survey pipelines.
  \item Tools \& Workflows ---\\
  -- Git, Jupyter, slurm (NERSC), LaTeX, Unix/Linux
\item Open-source Contributions ---\\
   -- \href{https://github.com/abhi0395/qsoabsfind}{qsoabsfind}: QSO metal absorber finder (\textit{Developer})\\
   -- \href{https://github.com/desihub/redrock}{redrock}: Redshift fitter for DESI (\textit{Contributor}).\\
   -- \href{https://github.com/desihub/desispec}{desispec}: DESI spectroscopic pipeline(\textit{Contributor}).
   %-- \href{https://github.com/desihub/{desihub}: Codebase for DESI Survey (\textit{Member}).
\end{list}

% \subsection{Awarded Supercomputing time}
% \begin{list}{}{\cvlist}
% \item -- 2024: Perlmutter Supercomputer at NERSC, \textit{Properties of         precious metal absorbers in DESI quasar spectra}, 1000 CPU hours, 1000 GPU      hours, PI: \textbf{A. Anand}
% \item -- 2023: Perlmutter Supercomputer at NERSC,  \textit{Archetype based      redshift estimation for DESI}, 1000 CPU hours, 1000 GPU hours, PI: \textbf{A.   Anand}
% \item -- 2018-2022: Freya Supercomputer at MPCDF,  \textit{The multiphase       galactic halo of galaxies and clusters with large spectroscopic and imaging     surveys}, 5000 CPU hours, PI: \textbf{A. Anand}
% \end{list}

%\subsection{Observing and data processing experience}
%\begin{list}{}{\cvlist}
%\item -- Sep 2022: Support Scientist for DESI spectroscopic pipeline
%\item -- Aug 2023: Remote Support Observing Scientist for DESI, Aug 12-14, 2023.
%\item -- Jul 2024: Support Observing Scientist for DESI, Kitt Peak National Observatory, Jul 2-5, 2024.
%\end{list}

\vspace{-2mm}
\subsection{Professional services \& activities}

\begin{list}{}{\cvlist}
    \item Referee --- \\
    -- Astronomy \& Astrophysics (2023-), Astrophysical Journal (2023-), Internal Reviewer for DESI papers (2022-)
    \item Member \& Organizer--- \\
    	-- Core committee member, \href{https://desi.lbl.gov/trac/wiki/PublicPages/Contacts#ProfessionalDevelopmentCommittee}{DESI Professional Development Committee}, (2024–25), contributed to early-career support and mentoring initiatives.\\
        --Organizer, \href{https://inpa.lbl.gov/events/}{INPA Weekly Seminar}, Physics Division, LBNL (2023-2024). \\
      \item Teaching Experience --- \\
        -- Served as online physics tutor at E-acharya (>20 students)\footnote{an initiative to help poor students in suburban and rural areas of Bihar, India.} from June 2017 to Aug 2018. \\
        \end{list}
  \vspace{-2mm}
\subsection{Outreach activities}

\begin{list}{}{\cvlist}
    \item Press releases --- \\
        -- My papers (\href{https://www.mpa-garching.mpg.de/964620/hl202107}{I}, \href{https://www.mpa-garching.mpg.de/1066558/hl202211?c=27981}{II}) featured in the monthly research highlight of MPA and appeared on \href{https://astrobites.org/2021/05/06/cool-metal-gas-search-thanks-it-was-automated/}{astrobites}.
    \item Interviews \& podcast --- \\
      -- Interviewed by \href{https://www.youtube.com/watch?v=WmA_PnYLeCg}{Vidpeds podcast} to discuss my astronomy journey and how people from small cities can become successful astronomers. \\
      -- Gave \href{https://theinterviewportal.com/2020/03/13/astrophysicist-interview-8/}{interview} to \href{https://theinterviewportal.com/}{The Interview Portal} to help Indian college students plan their careers.
  \end{list}

\ifdefined\withpubs
  \subsection{Publications}
  Total: 11 / refereed: 6 / first author: 3 / citations: 316 / h-index: 8 (2024-04-26)

  \subsubsection{First-author publications}
\begin{list}{}{\cvlist}
  \item[{\color{numcolor}\scriptsize5}] \textbf{Anand, Abhijeet}; Aguilar, J.; Ahlen, S.; Bianchi, D.; Brodzeller, A.; \etal, 2025, \doi{10.3847/1538-4357/adef3c}{The Cosmic Evolution of C IV Absorbers at 1.4 < z < 4.5: Insights from 100,000 Systems in DESI Quasars}, The Astrophysical Journal, \textbf{990}, 151 (\arxiv{2504.20299}) [\href{https://ui.adsabs.harvard.edu/abs/2025ApJ...990..151A}{2 citations}]

\item[{\color{numcolor}\scriptsize4}] \textbf{Anand, Abhijeet}; Guy, Julien; Bailey, Stephen; Moustakas, John; Aguilar, J.; \etal, 2024, \doi{10.3847/1538-3881/ad60c2}{Archetype-based Redshift Estimation for the Dark Energy Spectroscopic Instrument Survey}, The Astronomical Journal, \textbf{168}, 124 (\arxiv{2405.19288}) [\href{https://ui.adsabs.harvard.edu/abs/2024AJ....168..124A}{32 citations}]

\item[{\color{numcolor}\scriptsize3}] \textbf{Anand, Abhijeet}; Kauffmann, Guinevere; \& Nelson, Dylan, 2022, \doi{10.1093/mnras/stac928}{Cool circumgalactic gas in galaxy clusters: connecting the DESI legacy imaging survey and SDSS DR16 Mg II absorbers}, Monthly Notices of the Royal Astronomical Society, \textbf{513}, 3210 (\arxiv{2201.07811}) [\href{https://ui.adsabs.harvard.edu/abs/2022MNRAS.513.3210A}{28 citations}]

\item[{\color{numcolor}\scriptsize2}] \textbf{Anand, Abhijeet}; Nelson, Dylan; \& Kauffmann, Guinevere, 2021, \doi{10.1093/mnras/stab871}{Characterizing the abundance, properties, and kinematics of the cool circumgalactic medium of galaxies in absorption with SDSS DR16}, Monthly Notices of the Royal Astronomical Society, \textbf{504}, 65 (\arxiv{2103.15842}) [\href{https://ui.adsabs.harvard.edu/abs/2021MNRAS.504...65A}{55 citations}]

\item[{\color{numcolor}\scriptsize1}] \textbf{Anand, Abhijeet}; Roy, Nirupam; \& Gopal-Krishna, 2019, \doi{10.1088/1674-4527/19/6/83}{Search for H I emission from superdisk candidates associated with radio galaxies}, Research in Astronomy and Astrophysics, \textbf{19}, 083 (\arxiv{1812.06875}) [\href{https://ui.adsabs.harvard.edu/abs/2019RAA....19...83A}{2 citations}]
  %\newline \\
 %\input{in_prep}
\end{list}

\subsubsection{Significant contributions}
\begin{list}{}{\cvlist}
  \item[{\color{numcolor}\scriptsize5}] Wu, Xuanyi; Cai, Z.; Lan, T. -W.; Zou, S.; \textbf{Anand, Abhijeet}; \etal, 2025, \doi{10.3847/1538-4357/adb28a}{Tracing the Evolution of the Cool Gas in CGM and IGM Environments through Mg II Absorption from Redshift z = 0.75 to z = 1.65 Using DESI-Y1 Data}, The Astrophysical Journal, \textbf{983}, 186 (\arxiv{2407.17809}) [\href{https://ui.adsabs.harvard.edu/abs/2025ApJ...983..186W}{7 citations}]

\item[{\color{numcolor}\scriptsize4}] Chang, Yu-Ling; Lan, Ting-Wen; Prochaska, J. Xavier; Napolitano, Lucas; \textbf{Anand, Abhijeet}; \etal, 2024, \doi{10.3847/1538-4357/ad6c44}{Probing the Impact of Radio-mode Feedback on the Properties of the Cool Circumgalactic Medium}, The Astrophysical Journal, \textbf{974}, 191 (\arxiv{2405.08314}) [\href{https://ui.adsabs.harvard.edu/abs/2024ApJ...974..191C}{2 citations}]

\item[{\color{numcolor}\scriptsize3}] Galiullin, Ilkham; Rodriguez, Antonio C.; El-Badry, Kareem; Szkody, Paula; \textbf{Anand, Abhijeet}; \etal, 2024, \doi{10.1051/0004-6361/202450734}{Searching for new cataclysmic variables in the Chandra Source Catalog}, Astronomy and Astrophysics, \textbf{690} (\arxiv{2408.00078}) [\href{https://ui.adsabs.harvard.edu/abs/2024A&A...690A.374G}{4 citations}]

\item[{\color{numcolor}\scriptsize2}] Napolitano, Lucas; Pandey, Agnesh; Myers, Adam D.; Lan, Ting-Wen; \textbf{Anand, Abhijeet}; \etal, 2023, \doi{10.3847/1538-3881/ace62c}{Detecting and Characterizing Mg II Absorption in DESI Survey Validation Quasar Spectra}, The Astronomical Journal, \textbf{166}, 99 (\arxiv{2305.20016}) [\href{https://ui.adsabs.harvard.edu/abs/2023AJ....166...99N}{19 citations}]

\item[{\color{numcolor}\scriptsize1}] Ayromlou, Mohammadreza; Kauffmann, Guinevere; \textbf{Anand, Abhijeet}; \& White, Simon D. M., 2023, \doi{10.1093/mnras/stac3637}{The physical origin of galactic conformity: from theory to observation}, Monthly Notices of the Royal Astronomical Society, \textbf{519}, 1913 (\arxiv{2207.02218}) [\href{https://ui.adsabs.harvard.edu/abs/2023MNRAS.519.1913A}{24 citations}]
\end{list}

\subsubsection{Collaboration Papers}
\begin{list}{}{\cvlist}
  \item[{\color{numcolor}\scriptsize15}] Adame, A. G.; Aguilar, J.; Ahlen, S.; Alam, S.; Alexander, D. M.; \etal\ (incl.\ \textbf{A. Anand}), 2025, \doi{10.1088/1475-7516/2025/04/012}{DESI 2024 III: baryon acoustic oscillations from galaxies and quasars}, Journal of Cosmology and Astroparticle Physics, \textbf{2025}, 12 (\arxiv{2404.03000}) [\href{https://ui.adsabs.harvard.edu/abs/2025JCAP...04..012A}{274 citations}]

\item[{\color{numcolor}\scriptsize14}] DESI Collaboration; Abdul-Karim, M.; Aguilar, J.; Ahlen, S.; Alam, S.; \etal\ (incl.\ \textbf{A. Anand}), 2025, \doi{10.48550/arXiv.2503.14738}{DESI DR2 Results II: Measurements of Baryon Acoustic Oscillations and Cosmological Constraints}, ArXiv (\arxiv{2503.14738}) [\href{https://ui.adsabs.harvard.edu/abs/2025arXiv250314738D}{104 citations}]

\item[{\color{numcolor}\scriptsize13}] DESI Collaboration; Abdul-Karim, M.; Adame, A. G.; Aguado, D.; Aguilar, J.; \etal\ (incl.\ \textbf{A. Anand}), 2025, \doi{10.48550/arXiv.2503.14745}{Data Release 1 of the Dark Energy Spectroscopic Instrument}, ArXiv (\arxiv{2503.14745}) [\href{https://ui.adsabs.harvard.edu/abs/2025arXiv250314745D}{28 citations}]

\item[{\color{numcolor}\scriptsize12}] DESI Collaboration; Abdul-Karim, M.; Aguilar, J.; Ahlen, S.; Allende Prieto, C.; \etal\ (incl.\ \textbf{A. Anand}), 2025, \doi{10.48550/arXiv.2503.14739}{DESI DR2 Results I: Baryon Acoustic Oscillations from the Lyman Alpha Forest}, ArXiv (\arxiv{2503.14739}) [\href{https://ui.adsabs.harvard.edu/abs/2025arXiv250314739D}{38 citations}]

\item[{\color{numcolor}\scriptsize11}] Brodzeller, A.; Wolfson, M.; Santos, D. M.; Ho, M.; Tan, T.; \etal\ (incl.\ \textbf{A. Anand}), 2025, \doi{10.48550/arXiv.2503.14740}{Construction of the Damped Ly$\alpha$ Absorber Catalog for DESI DR2 Ly$\alpha$ BAO}, ArXiv (\arxiv{2503.14740}) [\href{https://ui.adsabs.harvard.edu/abs/2025arXiv250314740B}{6 citations}]

\item[{\color{numcolor}\scriptsize10}] Adame, A. G.; Aguilar, J.; Ahlen, S.; Alam, S.; Alexander, D. M.; \etal\ (incl.\ \textbf{A. Anand}), 2025, \doi{10.1088/1475-7516/2025/02/021}{DESI 2024 VI: cosmological constraints from the measurements of baryon acoustic oscillations}, Journal of Cosmology and Astroparticle Physics, \textbf{2025}, 21 (\arxiv{2404.03002}) [\href{https://ui.adsabs.harvard.edu/abs/2025JCAP...02..021A}{751 citations}]

\item[{\color{numcolor}\scriptsize9}] Adame, A. G.; Aguilar, J.; Ahlen, S.; Alam, S.; Alexander, D. M.; \etal\ (incl.\ \textbf{A. Anand}), 2025, \doi{10.1088/1475-7516/2025/01/124}{DESI 2024 IV: Baryon Acoustic Oscillations from the Lyman alpha forest}, Journal of Cosmology and Astroparticle Physics, \textbf{2025}, 124 (\arxiv{2404.03001}) [\href{https://ui.adsabs.harvard.edu/abs/2025JCAP...01..124A}{217 citations}]

\item[{\color{numcolor}\scriptsize8}] Ross, A. J.; Aguilar, J.; Ahlen, S.; Alam, S.; \textbf{Anand, Abhijeet}; \etal, 2025, \doi{10.1088/1475-7516/2025/01/125}{The construction of large-scale structure catalogs for the Dark Energy Spectroscopic Instrument}, Journal of Cosmology and Astroparticle Physics, \textbf{2025}, 125 (\arxiv{2405.16593}) [\href{https://ui.adsabs.harvard.edu/abs/2025JCAP...01..125R}{21 citations}]

\item[{\color{numcolor}\scriptsize7}] Scholte, Dirk; Saintonge, Am{\'e}lie; Moustakas, John; Catinella, Barbara; Zou, Hu; \etal\ (incl.\ \textbf{A. Anand}), 2024, \doi{10.1093/mnras/stae2477}{The atomic gas sequence and mass-metallicity relation from dwarfs to massive galaxies}, Monthly Notices of the Royal Astronomical Society, \textbf{535}, 2341 (\arxiv{2408.03996}) [\href{https://ui.adsabs.harvard.edu/abs/2024MNRAS.535.2341S}{6 citations}]

\item[{\color{numcolor}\scriptsize6}] DESI Collaboration; Adame, A. G.; Aguilar, J.; Ahlen, S.; Alam, S.; \etal\ (incl.\ \textbf{A. Anand}), 2024, \doi{10.48550/arXiv.2411.12021}{DESI 2024 V: Full-Shape Galaxy Clustering from Galaxies and Quasars}, ArXiv (\arxiv{2411.12021}) [\href{https://ui.adsabs.harvard.edu/abs/2024arXiv241112021D}{59 citations}]

\item[{\color{numcolor}\scriptsize5}] DESI Collaboration; Adame, A. G.; Aguilar, J.; Ahlen, S.; Alam, S.; \etal\ (incl.\ \textbf{A. Anand}), 2024, \doi{10.48550/arXiv.2411.12022}{DESI 2024 VII: Cosmological Constraints from the Full-Shape Modeling of Clustering Measurements}, ArXiv (\arxiv{2411.12022}) [\href{https://ui.adsabs.harvard.edu/abs/2024arXiv241112022D}{96 citations}]

\item[{\color{numcolor}\scriptsize4}] DESI Collaboration; Adame, A. G.; Aguilar, J.; Ahlen, S.; Alam, S.; \etal\ (incl.\ \textbf{A. Anand}), 2024, \doi{10.48550/arXiv.2411.12020}{DESI 2024 II: Sample Definitions, Characteristics, and Two-point Clustering Statistics}, ArXiv (\arxiv{2411.12020}) [\href{https://ui.adsabs.harvard.edu/abs/2024arXiv241112020D}{39 citations}]

\item[{\color{numcolor}\scriptsize3}] DESI Collaboration; Adame, A. G.; Aguilar, J.; Ahlen, S.; Alam, S.; \etal\ (incl.\ \textbf{A. Anand}), 2024, \doi{10.3847/1538-3881/ad3217}{The Early Data Release of the Dark Energy Spectroscopic Instrument}, The Astronomical Journal, \textbf{168}, 58 (\arxiv{2306.06308}) [\href{https://ui.adsabs.harvard.edu/abs/2024AJ....168...58D}{313 citations}]

\item[{\color{numcolor}\scriptsize2}] DESI Collaboration; Adame, A. G.; Aguilar, J.; Ahlen, S.; Alam, S.; \etal\ (incl.\ \textbf{A. Anand}), 2024, \doi{10.3847/1538-3881/ad0b08}{Validation of the Scientific Program for the Dark Energy Spectroscopic Instrument}, The Astronomical Journal, \textbf{167}, 62 (\arxiv{2306.06307}) [\href{https://ui.adsabs.harvard.edu/abs/2024AJ....167...62D}{189 citations}]

\item[{\color{numcolor}\scriptsize1}] Han, Jiwon Jesse; Dey, Arjun; Price-Whelan, Adrian M.; Najita, Joan; Schlafly, Edward F.; \etal\ (incl.\ \textbf{A. Anand}), 2023, \doi{10.48550/arXiv.2306.11784}{NANCY: Next-generation All-sky Near-infrared Community surveY}, ArXiv (\arxiv{2306.11784}) [\href{https://ui.adsabs.harvard.edu/abs/2023arXiv230611784H}{5 citations}]
\end{list}
\fi
\end{document}
