% The formatting of this CV is based on @davidwhogg's layout.

\documentclass[12pt,letterpaper]{article}

\usepackage{color}
\usepackage{fancyhdr}
\usepackage{hyperref}
\usepackage{ifthen}
\usepackage{enumitem}

% \usepackage[yyyymmdd]{datetime}
% \renewcommand{\dateseparator}{-}

% Link formatting.
\definecolor{numcolor}{rgb}{0.5,0.5,0.5}
\definecolor{linkcolor}{rgb}{0,0,0.4}
\hypersetup{%
    colorlinks=true,        % false: boxed links; true: colored links
    linkcolor=linkcolor,    % color of internal links
    citecolor=linkcolor,    % color of links to bibliography
    filecolor=linkcolor,    % color of file links
    urlcolor=linkcolor      % color of external links
}

% Text formatting.
\newcommand{\foreign}[1]{\textit{#1}}
\newcommand{\etal}{\foreign{et~al.}}
\newcommand{\project}[1]{\textsl{#1}}
\definecolor{grey}{rgb}{0.5,0.5,0.5}
\newcommand{\deemph}[1]{\textcolor{grey}{\footnotesize{#1}}}

% literature links--use doi if you can
  \newcommand{\doi}[2]{\emph{\href{http://dx.doi.org/#1}{{#2}}}}
  \newcommand{\ads}[2]{\href{http://adsabs.harvard.edu/abs/#1}{{#2}}}
  \newcommand{\isbn}[1]{{\footnotesize(\textsc{isbn:}{#1})}}
  \newcommand{\arxiv}[1]{{\href{http://arxiv.org/abs/#1}{arXiv:{#1}}}}

% Section headings.
\renewcommand\familydefault{\sfdefault}
\usepackage{titlesec}

\titleformat{\subsection}
{\normalfont\sffamily\large\bfseries}
{}{0pt}{}

\titleformat{\subsubsection}
{\normalfont\sffamily\bfseries}
{}{0pt}{}

\titlespacing{\subsection}{0pt}{2\parskip}{0pt}
\titlespacing{\subsubsection}{0pt}{\parskip}{0pt}

\newcommand{\cvheading}[1]{\addvspace{1ex}\pagebreak[2]\par\textbf{#1}\nopagebreak\vspace{-0.4em}}

% Set up the custom unordered list.
\newcounter{refpubnum}
\newcommand{\cvlist}{%
    \rightmargin=0in
    \leftmargin=0.15in
    \topsep=0ex
    \partopsep=0pt
    \itemsep=0.2ex
    \parsep=0pt
    \itemindent=-1.0\leftmargin
    \listparindent=0.0\leftmargin
    \settowidth{\labelsep}{~}
    \usecounter{refpubnum}
}

% Margins and spaces.
\raggedright
\setlength{\oddsidemargin}{0in}
\setlength{\topmargin}{0in}
\setlength{\headsep}{0.20in}
\setlength{\headheight}{0.25in}
\setlength{\textheight}{9.1in}
\addtolength{\topmargin}{-\headsep}
\addtolength{\topmargin}{-\headheight}
\setlength{\textwidth}{6.50in}
\setlength{\parindent}{0in}
\setlength{\parskip}{1ex}

% Headings and footings.
\renewcommand{\headrulewidth}{0pt}
\pagestyle{fancy}
\lhead{\deemph{Abhijeet Anand}}
\chead{\deemph{Curriculum Vitae}}
\rhead{\deemph{\thepage}}
\cfoot{\deemph{Last updated: \today}}

% Journal names.
\newcommand{\aj}{AJ}
\newcommand{\apj}{ApJ}
\newcommand{\pasp}{PASP}
\newcommand{\mnras}{MNRAS}


\begin{document}\thispagestyle{empty}\sloppy\sloppypar\raggedbottom

\textbf{\Large Abhijeet Anand} \hfill
\textsf{\small https://abhi0395.github.io} \\[0.5ex]
Postdoctoral Scientist, Lawrence Berkeley National Lab, CA, USA\\[0.5ex]

\subsection{Education}
\begin{list}{}{\cvlist}
  \item
        PhD 2022, Faculty of Physics, LMU Munich - Max Planck Institute for Astrophysics. Advisor: Prof.Dr. Guinevere Kauffmann \& Dr. Dylan Nelson
  \item
        MS 2017, Department of Physics, Indian Institute of Science, Bangalore, Advisor: Dr. Nirupam Roy
  \item
        BS 2016, Department of Physics, Indian Institute of Science, Bangalore, Advisor: Prof. Gajendra Pandey, IIA
\end{list}

\subsection{Positions}
\begin{list}{}{\cvlist}
  \item
      Postdoctoral Scientist, \href{https://www.desi.lbl.gov/}{Dark Energy Survey Instrument (DESI) Group} Lawrence Berkeley National Lab, Berkeley, Sep 2022--present.
  \item
        Postdoctoral Fellow, Max Planck Institute for Astrophysics (MPA), Germany, Aug 2022
  \item
        Junior Research Fellow (JRF), National Institute of Advanced Studies, Bangalore, Sep 2017 - Jul 2018.
\end{list}

\ifdefined\withpubs
  \subsection{Publications}
  Total: 25 / refereed: 18 / first author: 5 / citations: 2,929 / h-index: 17 (Last updated: 2025-07-29), List attached below

  \subsubsection{Refereed publications}
  \begin{list}{}{\cvlist}
    \item[{\color{numcolor}\scriptsize18}] Adame, A. G.; Aguilar, J.; Ahlen, S.; Alam, S.; Alexander, D. M.; \etal\ (incl.\ \textbf{A. Anand}), 2025, \doi{10.1088/1475-7516/2025/07/028}{DESI 2024 VII: cosmological constraints from the full-shape modeling of clustering measurements}, Journal of Cosmology and Astroparticle Physics, \textbf{2025}, 28 (\arxiv{2411.12022}) [\href{https://ui.adsabs.harvard.edu/abs/2025JCAP...07..028A}{136 citations}]

\item[{\color{numcolor}\scriptsize17}] Adame, A. G.; Aguilar, J.; Ahlen, S.; Alam, S.; Alexander, D. M.; \etal\ (incl.\ \textbf{A. Anand}), 2025, \doi{10.1088/1475-7516/2025/07/017}{DESI 2024 II: sample definitions, characteristics, and two-point clustering statistics}, Journal of Cosmology and Astroparticle Physics, \textbf{2025}, 17 (\arxiv{2411.12020}) [\href{https://ui.adsabs.harvard.edu/abs/2025JCAP...07..017A}{53 citations}]

\item[{\color{numcolor}\scriptsize16}] Adame, A. G.; Aguilar, J.; Ahlen, S.; Alam, S.; Alexander, D. M.; \etal\ (incl.\ \textbf{A. Anand}), 2025, \doi{10.1088/1475-7516/2025/04/012}{DESI 2024 III: baryon acoustic oscillations from galaxies and quasars}, Journal of Cosmology and Astroparticle Physics, \textbf{2025}, 12 (\arxiv{2404.03000}) [\href{https://ui.adsabs.harvard.edu/abs/2025JCAP...04..012A}{314 citations}]

\item[{\color{numcolor}\scriptsize15}] Wu, Xuanyi; Cai, Z.; Lan, T. -W.; Zou, S.; \textbf{Anand, Abhijeet}; \etal, 2025, \doi{10.3847/1538-4357/adb28a}{Tracing the Evolution of the Cool Gas in CGM and IGM Environments through Mg II Absorption from Redshift z = 0.75 to z = 1.65 Using DESI-Y1 Data}, The Astrophysical Journal, \textbf{983}, 186 (\arxiv{2407.17809}) [\href{https://ui.adsabs.harvard.edu/abs/2025ApJ...983..186W}{6 citations}]

\item[{\color{numcolor}\scriptsize14}] Adame, A. G.; Aguilar, J.; Ahlen, S.; Alam, S.; Alexander, D. M.; \etal\ (incl.\ \textbf{A. Anand}), 2025, \doi{10.1088/1475-7516/2025/02/021}{DESI 2024 VI: cosmological constraints from the measurements of baryon acoustic oscillations}, Journal of Cosmology and Astroparticle Physics, \textbf{2025}, 21 (\arxiv{2404.03002}) [\href{https://ui.adsabs.harvard.edu/abs/2025JCAP...02..021A}{919 citations}]

\item[{\color{numcolor}\scriptsize13}] Adame, A. G.; Aguilar, J.; Ahlen, S.; Alam, S.; Alexander, D. M.; \etal\ (incl.\ \textbf{A. Anand}), 2025, \doi{10.1088/1475-7516/2025/01/124}{DESI 2024 IV: Baryon Acoustic Oscillations from the Lyman alpha forest}, Journal of Cosmology and Astroparticle Physics, \textbf{2025}, 124 (\arxiv{2404.03001}) [\href{https://ui.adsabs.harvard.edu/abs/2025JCAP...01..124A}{252 citations}]

\item[{\color{numcolor}\scriptsize12}] Ross, A. J.; Aguilar, J.; Ahlen, S.; Alam, S.; \textbf{Anand, Abhijeet}; \etal, 2025, \doi{10.1088/1475-7516/2025/01/125}{The construction of large-scale structure catalogs for the Dark Energy Spectroscopic Instrument}, Journal of Cosmology and Astroparticle Physics, \textbf{2025}, 125 (\arxiv{2405.16593}) [\href{https://ui.adsabs.harvard.edu/abs/2025JCAP...01..125R}{26 citations}]

\item[{\color{numcolor}\scriptsize11}] Scholte, Dirk; Saintonge, Am{\'e}lie; Moustakas, John; Catinella, Barbara; Zou, Hu; \etal\ (incl.\ \textbf{A. Anand}), 2024, \doi{10.1093/mnras/stae2477}{The atomic gas sequence and mass-metallicity relation from dwarfs to massive galaxies}, Monthly Notices of the Royal Astronomical Society, \textbf{535}, 2341 (\arxiv{2408.03996}) [\href{https://ui.adsabs.harvard.edu/abs/2024MNRAS.535.2341S}{8 citations}]

\item[{\color{numcolor}\scriptsize10}] Chang, Yu-Ling; Lan, Ting-Wen; Prochaska, J. Xavier; Napolitano, Lucas; \textbf{Anand, Abhijeet}; \etal, 2024, \doi{10.3847/1538-4357/ad6c44}{Probing the Impact of Radio-mode Feedback on the Properties of the Cool Circumgalactic Medium}, The Astrophysical Journal, \textbf{974}, 191 (\arxiv{2405.08314}) [\href{https://ui.adsabs.harvard.edu/abs/2024ApJ...974..191C}{2 citations}]

\item[{\color{numcolor}\scriptsize9}] Galiullin, Ilkham; Rodriguez, Antonio C.; El-Badry, Kareem; Szkody, Paula; \textbf{Anand, Abhijeet}; \etal, 2024, \doi{10.1051/0004-6361/202450734}{Searching for new cataclysmic variables in the Chandra Source Catalog}, Astronomy and Astrophysics, \textbf{690} (\arxiv{2408.00078}) [\href{https://ui.adsabs.harvard.edu/abs/2024A&A...690A.374G}{4 citations}]

\item[{\color{numcolor}\scriptsize8}] \textbf{Anand, Abhijeet}; Guy, Julien; Bailey, Stephen; Moustakas, John; Aguilar, J.; \etal, 2024, \doi{10.3847/1538-3881/ad60c2}{Archetype-based Redshift Estimation for the Dark Energy Spectroscopic Instrument Survey}, The Astronomical Journal, \textbf{168}, 124 (\arxiv{2405.19288}) [\href{https://ui.adsabs.harvard.edu/abs/2024AJ....168..124A}{20 citations}]

\item[{\color{numcolor}\scriptsize7}] DESI Collaboration; Adame, A. G.; Aguilar, J.; Ahlen, S.; Alam, S.; \etal\ (incl.\ \textbf{A. Anand}), 2024, \doi{10.3847/1538-3881/ad3217}{The Early Data Release of the Dark Energy Spectroscopic Instrument}, The Astronomical Journal, \textbf{168}, 58 (\arxiv{2306.06308}) [\href{https://ui.adsabs.harvard.edu/abs/2024AJ....168...58D}{359 citations}]

\item[{\color{numcolor}\scriptsize6}] DESI Collaboration; Adame, A. G.; Aguilar, J.; Ahlen, S.; Alam, S.; \etal\ (incl.\ \textbf{A. Anand}), 2024, \doi{10.3847/1538-3881/ad0b08}{Validation of the Scientific Program for the Dark Energy Spectroscopic Instrument}, The Astronomical Journal, \textbf{167}, 62 (\arxiv{2306.06307}) [\href{https://ui.adsabs.harvard.edu/abs/2024AJ....167...62D}{201 citations}]

\item[{\color{numcolor}\scriptsize5}] Napolitano, Lucas; Pandey, Agnesh; Myers, Adam D.; Lan, Ting-Wen; \textbf{Anand, Abhijeet}; \etal, 2023, \doi{10.3847/1538-3881/ace62c}{Detecting and Characterizing Mg II Absorption in DESI Survey Validation Quasar Spectra}, The Astronomical Journal, \textbf{166}, 99 (\arxiv{2305.20016}) [\href{https://ui.adsabs.harvard.edu/abs/2023AJ....166...99N}{19 citations}]

\item[{\color{numcolor}\scriptsize4}] Ayromlou, Mohammadreza; Kauffmann, Guinevere; \textbf{Anand, Abhijeet}; \& White, Simon D. M., 2023, \doi{10.1093/mnras/stac3637}{The physical origin of galactic conformity: from theory to observation}, Monthly Notices of the Royal Astronomical Society, \textbf{519}, 1913 (\arxiv{2207.02218}) [\href{https://ui.adsabs.harvard.edu/abs/2023MNRAS.519.1913A}{24 citations}]

\item[{\color{numcolor}\scriptsize3}] \textbf{Anand, Abhijeet}; Kauffmann, Guinevere; \& Nelson, Dylan, 2022, \doi{10.1093/mnras/stac928}{Cool circumgalactic gas in galaxy clusters: connecting the DESI legacy imaging survey and SDSS DR16 Mg II absorbers}, Monthly Notices of the Royal Astronomical Society, \textbf{513}, 3210 (\arxiv{2201.07811}) [\href{https://ui.adsabs.harvard.edu/abs/2022MNRAS.513.3210A}{26 citations}]

\item[{\color{numcolor}\scriptsize2}] \textbf{Anand, Abhijeet}; Nelson, Dylan; \& Kauffmann, Guinevere, 2021, \doi{10.1093/mnras/stab871}{Characterizing the abundance, properties, and kinematics of the cool circumgalactic medium of galaxies in absorption with SDSS DR16}, Monthly Notices of the Royal Astronomical Society, \textbf{504}, 65 (\arxiv{2103.15842}) [\href{https://ui.adsabs.harvard.edu/abs/2021MNRAS.504...65A}{44 citations}]

\item[{\color{numcolor}\scriptsize1}] \textbf{Anand, Abhijeet}; Roy, Nirupam; \& Gopal-Krishna, 2019, \doi{10.1088/1674-4527/19/6/83}{Search for H I emission from superdisk candidates associated with radio galaxies}, Research in Astronomy and Astrophysics, \textbf{19}, 83 (\arxiv{1812.06875}) [\href{https://ui.adsabs.harvard.edu/abs/2019RAA....19...83A}{2 citations}]
  \end{list}

  \subsubsection{Preprints \& white papers}
  \begin{list}{}{\cvlist}
    \item[{\color{numcolor}\scriptsize8}] Medina, Gustavo E.; Li, Ting S.; Eadie, Gwendolyn M.; Riley, Alexander H.; Valluri, Monica; \etal\ (incl.\ \textbf{A. Anand}), 2025, \doi{10.48550/arXiv.2508.19351}{The mass of the Milky Way from outer halo stars measured by DESI DR1}, ArXiv, arXiv:2508.19351 (\arxiv{2508.19351})

\item[{\color{numcolor}\scriptsize7}] Rashkovetskyi, M.; Eisenstein, D. J.; Aguilar, J.; Ahlen, S.; \textbf{Anand, Abhijeet}; \etal, 2025, \doi{10.48550/arXiv.2508.20904}{Clustering of DESI galaxies split by thermal Sunyaev-Zeldovich effect}, ArXiv, arXiv:2508.20904 (\arxiv{2508.20904})

\item[{\color{numcolor}\scriptsize6}] Herrera-Alcantar, Hiram K.; Armengaud, Eric; Y{\`e}che, Christophe; Gordon, Calum; Casas, Laura; \etal\ (incl.\ \textbf{A. Anand}), 2025, \doi{10.48550/arXiv.2507.21852}{The Lyman-${\ensuremath{\alpha}}$ Forest from LBGs: First 3D Correlation Measurement with DESI and Prospects for Cosmology}, ArXiv, arXiv:2507.21852 (\arxiv{2507.21852}) [\href{https://ui.adsabs.harvard.edu/abs/2025arXiv250721852H}{1 citations}]

\item[{\color{numcolor}\scriptsize5}] DESI Collaboration; Abdul-Karim, M.; Aguilar, J.; Ahlen, S.; Alam, S.; \etal\ (incl.\ \textbf{A. Anand}), 2025, \doi{10.48550/arXiv.2503.14738}{DESI DR2 Results II: Measurements of Baryon Acoustic Oscillations and Cosmological Constraints}, ArXiv, arXiv:2503.14738 (\arxiv{2503.14738}) [\href{https://ui.adsabs.harvard.edu/abs/2025arXiv250314738D}{386 citations}]

\item[{\color{numcolor}\scriptsize4}] DESI Collaboration; Abdul-Karim, M.; Adame, A. G.; Aguado, D.; Aguilar, J.; \etal\ (incl.\ \textbf{A. Anand}), 2025, \doi{10.48550/arXiv.2503.14745}{Data Release 1 of the Dark Energy Spectroscopic Instrument}, ArXiv, arXiv:2503.14745 (\arxiv{2503.14745}) [\href{https://ui.adsabs.harvard.edu/abs/2025arXiv250314745D}{116 citations}]

\item[{\color{numcolor}\scriptsize3}] DESI Collaboration; Abdul-Karim, M.; Aguilar, J.; Ahlen, S.; Allende Prieto, C.; \etal\ (incl.\ \textbf{A. Anand}), 2025, \doi{10.48550/arXiv.2503.14739}{DESI DR2 Results I: Baryon Acoustic Oscillations from the Lyman Alpha Forest}, ArXiv, arXiv:2503.14739 (\arxiv{2503.14739}) [\href{https://ui.adsabs.harvard.edu/abs/2025arXiv250314739D}{78 citations}]

\item[{\color{numcolor}\scriptsize2}] Brodzeller, A.; Wolfson, M.; Santos, D. M.; Ho, M.; Tan, T.; \etal\ (incl.\ \textbf{A. Anand}), 2025, \doi{10.48550/arXiv.2503.14740}{Construction of the Damped Ly${\ensuremath{\alpha}}$ Absorber Catalog for DESI DR2 Ly${\ensuremath{\alpha}}$ BAO}, ArXiv, arXiv:2503.14740 (\arxiv{2503.14740}) [\href{https://ui.adsabs.harvard.edu/abs/2025arXiv250314740B}{11 citations}]

\item[{\color{numcolor}\scriptsize1}] Han, Jiwon Jesse; Dey, Arjun; Price-Whelan, Adrian M.; Najita, Joan; Schlafly, Edward F.; \etal\ (incl.\ \textbf{A. Anand}), 2023, \doi{10.48550/arXiv.2306.11784}{NANCY: Next-generation All-sky Near-infrared Community surveY}, ArXiv, arXiv:2306.11784 (\arxiv{2306.11784}) [\href{https://ui.adsabs.harvard.edu/abs/2023arXiv230611784H}{6 citations}]
  \end{list}
\fi

\subsection{Selected invited talks \& tutorials}
\begin{list}{}{\cvlist}

  \item \emph{Open software for Astrophysics},
      2023, Invited Plenary, 241st AAS Meeting, Seattle.

  \item \emph{Gaussian Processes for EPRV},
      2022, Invited Tutorial, University of Oxford, UK.

  \item \emph{Methods for scalable probabilistic inference},
      2022, Colloquium, University of Illinois Urbana-Champaign.\\
      2022, Colloquium, UC Berkeley.\\
      2022, Colloquium, University of Oxford, UK.\\
      2021, Invited Talk, Institute for Pure \& Applied Mathematics, UCLA.

  \item \emph{Advanced probabilistic modeling},
      2021, Tutorial, Harley Wood Winter School of Astronomy, Australia.

  \item \emph{Open-source software for probabilistic data analysis in astronomy},
      2021, Seminar, Instituto de Astrof\'isica, Portugal.

  \item \emph{Gaussian processes \& stellar variability},
      2021, Seminar, CARMENES Team Meeting.

  \item \emph{Extending JAX with custom C++ \& CUDA},
        2021, Invited Talk, IRIS-HEP Topical Meeting, CERN.

  \item \emph{Open source software for probabilistic data analysis},
        2020, Invited Talk, OzGrav Early Career Researcher Symposium, Australia.

  \item \emph{The why \& how of exoplanet, a domain-specific PyMC3 extension},
        2020, Contributed Talk, PyMC Con.

  \item \emph{A modular ecosystem for probabilistic data analysis},
        2019, Invited Talk, Open Digital Infrastructure in Astronomy conference,
        Kavli Institute for Theoretical Physics.

  \item \emph{Exoplanet population inference, a tutorial},
        2019, Invited Talk, Exostar19 conference,
        Kavli Institute for Theoretical Physics.

  \item \emph{Astronomy as a testbed for statistical method development},
        2019, Colloquium, Center for Statistics and Machine Learning,
        Princeton.

  \item \emph{Data-driven discovery in the astronomical time domain},
        2018, Colloquium, Institute for Theory and Computation,
        Harvard-Smithsonian Center for Astrophysics.\\
        2018, Colloquium, University of California, Santa Cruz.\\
        2017, Interdisciplinary Colloquium, CIERA, Northwestern University.

  \item \emph{A practical introduction to Gaussian Processes for astronomy},
        2017, Invited Talk, Statistical Challenges in Astrophysics,
        University of New South Wales, Australia.

  \item \emph{Long-period transiting planets \& their population},
        2016, Invited talk, Exoplanets I, Davos. \\
        2016, Invited talk, Statistical Challenges of Modern Astrophysics,
        Carnegie Mellon.\\
        2016, Colloquium, Villanova.

  \item \emph{Scalable Gaussian processes \& the search for transiting
          exoplanets}, 2015, Data Science at the LHC, CERN, Geneva.

  \item \emph{Discovery \& characterization of transiting exoplanets \& their
          population}, 2015, Colloquium, University of Washington.

  \item \emph{Hierarchical inference for exoplanet population inference},
        2015, IAU Symposium, Honolulu.

  \item \emph{Data-driven models}, 2015, Extreme precision radial velocities,
        Yale.

  \item \emph{Population inference from noisy \& incomplete catalogs}, 2015,
        Local Group Astrostatistics, University of Michigan.

  \item \emph{Time series analysis, Gaussian Processes, and the search for
          exo-Earths},
        2014, PyData NYC conference, New York.

  \item \emph{Introduction to Gaussian Processes, probabilistic graphical
          models, and deep learning},
        2014, Astro Hack Week, University of Washington.

  \item \emph{An astronomer's introduction to Gaussian processes},
        2014, Bayesian Computing for Astronomical Data Analysis (Summer school at
        Penn State University).

\end{list}

\subsection{Grants}
\begin{list}{}{\cvlist}
  \item NSF-CDS\&E (PI: Agol)
        \emph{Development of fast, multi-dimensional Gaussian Processes for Exoplanet discovery and beyond},
        \$471,048.00, 2019--2022

  \item
        NSF-AAG (PI: Agol),
        \emph{Collaborative Research: Masses and architectures of (potentially
          habitable) exoplanet systems},
        \$491,950, 2016--2018

  \item
        K2 Guest Observer -- Cycle 3 (PI: Penny),
        \emph{Free-Floating and Bound Planet Mass Measurements with K2: Ground- and
          Space-Based Photometry, Event Detection and Modeling},
        \$84,000, 2016--2017

  \item
        K2 Guest Observer -- Cycle 3 (PI: Hogg),
        \emph{Ultra-precise photometry in crowded fields: A self-calibration
          approach},
        \$100,000, 2016--2017

  \item
        XSEDE (PI: Foreman-Mackey),
        \emph{A systematic search for transiting exoplanets using K2},
        100,000 CPU hours, 2015--2016
\end{list}


\subsection{Honors}
\begin{list}{}{\cvlist}

  \item Kavli Fellow, 2015.
  \item Sagan Postdoctoral Fellowship, 2015--2017.
  \item James Arthur Graduate Fellowship, 2014.
  \item Horizon Fellowship in the Natural \& Physical Sciences, 2012.
  \item Henry M. MacCracken Fellowship, 2010.
  \item NSERC Undergraduate Summer Research Award, 2007.

\end{list}

% \ifdefined\withpubs
%     \newpage
% \fi

\subsection{Professional service \& activities}
\begin{list}{}{\cvlist}
  \item Associate Editor-in-Chief --- Journal of Open Source Software
  \item Active Referee ---
        AAS Journals,
        MNRAS,
        PASP,
        A\&A,
        Journal of Statistical Software,
        Journal on Uncertainty Quantification,
        Journal of Open Source Software
  \item Panelist ---
        NSF, NASA, LSSTC
\end{list}

\end{document}
