% The formatting of this CV is based on @davidwhogg's layout.

\documentclass[12pt,letterpaper]{article}
\usepackage{color}
\usepackage{fancyhdr}
\usepackage{hyperref}
\usepackage{ifthen}
\usepackage{enumitem}

% \usepackage[yyyymmdd]{datetime}
% \renewcommand{\dateseparator}{-}

% Link formatting.
\definecolor{numcolor}{rgb}{0.5,0.5,0.5}
\definecolor{linkcolor}{rgb}{0,0,0.4}
\hypersetup{%
    colorlinks=true,        % false: boxed links; true: colored links
    linkcolor=linkcolor,    % color of internal links
    citecolor=linkcolor,    % color of links to bibliography
    filecolor=linkcolor,    % color of file links
    urlcolor=linkcolor      % color of external links
}

% Text formatting.
\newcommand{\foreign}[1]{\textit{#1}}
\newcommand{\etal}{\foreign{et~al.}}
\newcommand{\project}[1]{\textsl{#1}}
\definecolor{grey}{rgb}{0.5,0.5,0.5}
\newcommand{\deemph}[1]{\textcolor{grey}{\footnotesize{#1}}}

% literature links--use doi if you can
  \newcommand{\doi}[2]{\emph{\href{http://dx.doi.org/#1}{{#2}}}}
  \newcommand{\ads}[2]{\href{http://adsabs.harvard.edu/abs/#1}{{#2}}}
  \newcommand{\isbn}[1]{{\footnotesize(\textsc{isbn:}{#1})}}
  \newcommand{\arxiv}[1]{{\href{http://arxiv.org/abs/#1}{arXiv:{#1}}}}

% Section headings.
\renewcommand\familydefault{\sfdefault}
\usepackage{titlesec}

\titleformat{\subsection}
{\normalfont\sffamily\large\bfseries}
{}{0pt}{}

\titleformat{\subsubsection}
{\normalfont\sffamily\bfseries}
{}{0pt}{}

\titlespacing{\subsection}{0pt}{2\parskip}{0pt}
\titlespacing{\subsubsection}{0pt}{\parskip}{0pt}

\newcommand{\cvheading}[1]{\addvspace{1ex}\pagebreak[2]\par\textbf{#1}\nopagebreak\vspace{-0.4em}}

% Set up the custom unordered list.
\newcounter{refpubnum}
\newcommand{\cvlist}{%
    \rightmargin=0in
    \leftmargin=0.15in
    \topsep=0ex
    \partopsep=0pt
    \itemsep=0.2ex
    \parsep=0pt
    \itemindent=-1.0\leftmargin
    \listparindent=0.0\leftmargin
    \settowidth{\labelsep}{~}
    \usecounter{refpubnum}
}

% Margins and spaces.
\raggedright
\setlength{\oddsidemargin}{0in}
\setlength{\topmargin}{0in}
\setlength{\headsep}{0.20in}
\setlength{\headheight}{0.25in}
\setlength{\textheight}{9.1in}
\addtolength{\topmargin}{-\headsep}
\addtolength{\topmargin}{-\headheight}
\setlength{\textwidth}{6.50in}
\setlength{\parindent}{0in}
\setlength{\parskip}{1ex}

% Headings and footings.
\renewcommand{\headrulewidth}{0pt}
\pagestyle{fancy}
\lhead{\deemph{Abhijeet Anand}}
\chead{\deemph{Curriculum Vitae}}
\rhead{\deemph{\thepage}}
\cfoot{\deemph{Last updated: \today}}

% Journal names.
\newcommand{\aj}{AJ}
\newcommand{\apj}{ApJ}
\newcommand{\pasp}{PASP}
\newcommand{\mnras}{MNRAS}

\newcommand{\tex}[1]{#1}

\begin{document}\thispagestyle{empty}\sloppy\sloppypar\raggedbottom

\textbf{\Large Abhijeet Anand, PhD}\\[0.5ex]
Postdoctoral Scientist, Lawrence Berkeley National Lab, CA, USA\\[0.5ex]
\textsf{\small AbhijeetAnand@lbl.gov, \href{https://abhi0395.github.io/}{https://abhi0395.github.io/}}\\[0.5ex]

\subsection{Research Interest}
My research primarily revolves around investigating gas flows within the galaxy halo, commonly referred to as the circumgalactic medium, using quasar absorption lines. I explore the connection between these gas flows and the formation and evolution of galaxies across different epochs. Furthermore, I am deeply interested in developing and learning innovative methodologies for fitting and analysing astronomical spectra coming from large spectroscopic surveys. Additionally, fostering a healthy environment where individuals from diverse backgrounds and perspectives can thrive collectively is a core aspect of my professional values.

\subsection{Positions}
\begin{list}{}{\cvlist}
\item Sep'22 - present: Postdoctoral Scientist, Lawrence Berkeley National Lab, Berkeley, CA, USA \\
%\begin{itemize}
%  \item Group: \href{https://www.desi.lbl.gov/}{Dark Energy Survey Instrument (DESI)}, \emph{Supervisor}: Dr. Julien Guy
%  \item \emph{Improving redshift estimation algorithm for DESI galaxies, using DESI spectra to characterize cool/warm gas in galactic halo}
%\end{itemize}
\item Aug'22: Bridge Postdoctoral Fellow, Max Planck Institute for Astrophysics, Garching, Germany
\item Sep'18 - Jul'22: IMPRS PhD Fellow, Max Planck Institute for Astrophysics, Garching, Germany
\item Sep'17 - Jul'18: UGC - JRF\footnote{University Grants Commission, Govt. of India - Junior Research Fellowship}, National Institute of Advanced Studies, Bangalore, India 
\end{list}

\subsection{Education}
\begin{list}{}{\cvlist}
  \item Sep 2018 - Jul 2022: PhD in Astrophysics, Ludwig Maximilian University - Max Planck Institute for Astrophysics (MPA), Germany.
\begin{itemize}
    \item Thesis: \href{https://edoc.ub.uni-muenchen.de/30337/}{Probing cool and warm circumgalactic gas in galaxies and clusters with large spectroscopic and imaging surveys}
    \item Advisors: \href{https://www.mpa-garching.mpg.de/galaxyformation}{Prof. Dr. Guinevere Kauffmann} \& \href{https://www.ita.uni-heidelberg.de/~dnelson/}{Dr. Dylan Nelson}
\end{itemize}

\item Jul 2016 - Jul 2017: MS, Physics, Indian Institute of Science (IISc), Bangalore, India.
\begin{itemize}
    \item Thesis: \href{https://raw.githubusercontent.com/abhi0395/mycv/main/files/MS_thesis.pdf}{A sensitive search for HI 21 cm emission from super disks in radio galaxies}
    \item Advisor: \href{http://www.physics.iisc.ernet.in/%7Enroy/}{Prof. Nirupam Roy}
  \end{itemize}
\item Aug 2012 - June 2016: BSc (Research), Physics, Indian Institute of Science, Bangalore, India.
\begin{itemize}
    \item Thesis: \href{https://raw.githubusercontent.com/abhi0395/mycv/main/files/BS_thesis.pdf}{Sources of Continuum Opacity in Hydrogen deficient stars}
    \item Advisors: \href{https://www.iiap.res.in/?q=user/29}{Prof. Gajendra Pandey}, \href{https://www.iiap.res.in/}{Indian Institute of Astrophysics (IIA)}
  \end{itemize}
\end{list}

%\subsection{Research Interest}
%\begin{itemize}
%    \item Understanding gas in flows in galaxy halo (aka circumgalactic medium) using quasar absorption lines and its connection to galaxy formation and evolution
%    \item Developing new tools to fit and analyze astronomical spectra coming from large spectrscopic   surveys
%\end{itemize}

\ifdefined\withpubs
  \subsection{Publications}
  Total: 25 / refereed: 18 / first author: 5 / citations: 2,929 / h-index: 17 (Last updated: 2025-07-29), List attached below

  \subsubsection{Refereed publications}
  \begin{list}{}{\cvlist}
    \item[{\color{numcolor}\scriptsize18}] Adame, A. G.; Aguilar, J.; Ahlen, S.; Alam, S.; Alexander, D. M.; \etal\ (incl.\ \textbf{A. Anand}), 2025, \doi{10.1088/1475-7516/2025/07/028}{DESI 2024 VII: cosmological constraints from the full-shape modeling of clustering measurements}, Journal of Cosmology and Astroparticle Physics, \textbf{2025}, 28 (\arxiv{2411.12022}) [\href{https://ui.adsabs.harvard.edu/abs/2025JCAP...07..028A}{136 citations}]

\item[{\color{numcolor}\scriptsize17}] Adame, A. G.; Aguilar, J.; Ahlen, S.; Alam, S.; Alexander, D. M.; \etal\ (incl.\ \textbf{A. Anand}), 2025, \doi{10.1088/1475-7516/2025/07/017}{DESI 2024 II: sample definitions, characteristics, and two-point clustering statistics}, Journal of Cosmology and Astroparticle Physics, \textbf{2025}, 17 (\arxiv{2411.12020}) [\href{https://ui.adsabs.harvard.edu/abs/2025JCAP...07..017A}{53 citations}]

\item[{\color{numcolor}\scriptsize16}] Adame, A. G.; Aguilar, J.; Ahlen, S.; Alam, S.; Alexander, D. M.; \etal\ (incl.\ \textbf{A. Anand}), 2025, \doi{10.1088/1475-7516/2025/04/012}{DESI 2024 III: baryon acoustic oscillations from galaxies and quasars}, Journal of Cosmology and Astroparticle Physics, \textbf{2025}, 12 (\arxiv{2404.03000}) [\href{https://ui.adsabs.harvard.edu/abs/2025JCAP...04..012A}{314 citations}]

\item[{\color{numcolor}\scriptsize15}] Wu, Xuanyi; Cai, Z.; Lan, T. -W.; Zou, S.; \textbf{Anand, Abhijeet}; \etal, 2025, \doi{10.3847/1538-4357/adb28a}{Tracing the Evolution of the Cool Gas in CGM and IGM Environments through Mg II Absorption from Redshift z = 0.75 to z = 1.65 Using DESI-Y1 Data}, The Astrophysical Journal, \textbf{983}, 186 (\arxiv{2407.17809}) [\href{https://ui.adsabs.harvard.edu/abs/2025ApJ...983..186W}{6 citations}]

\item[{\color{numcolor}\scriptsize14}] Adame, A. G.; Aguilar, J.; Ahlen, S.; Alam, S.; Alexander, D. M.; \etal\ (incl.\ \textbf{A. Anand}), 2025, \doi{10.1088/1475-7516/2025/02/021}{DESI 2024 VI: cosmological constraints from the measurements of baryon acoustic oscillations}, Journal of Cosmology and Astroparticle Physics, \textbf{2025}, 21 (\arxiv{2404.03002}) [\href{https://ui.adsabs.harvard.edu/abs/2025JCAP...02..021A}{919 citations}]

\item[{\color{numcolor}\scriptsize13}] Adame, A. G.; Aguilar, J.; Ahlen, S.; Alam, S.; Alexander, D. M.; \etal\ (incl.\ \textbf{A. Anand}), 2025, \doi{10.1088/1475-7516/2025/01/124}{DESI 2024 IV: Baryon Acoustic Oscillations from the Lyman alpha forest}, Journal of Cosmology and Astroparticle Physics, \textbf{2025}, 124 (\arxiv{2404.03001}) [\href{https://ui.adsabs.harvard.edu/abs/2025JCAP...01..124A}{252 citations}]

\item[{\color{numcolor}\scriptsize12}] Ross, A. J.; Aguilar, J.; Ahlen, S.; Alam, S.; \textbf{Anand, Abhijeet}; \etal, 2025, \doi{10.1088/1475-7516/2025/01/125}{The construction of large-scale structure catalogs for the Dark Energy Spectroscopic Instrument}, Journal of Cosmology and Astroparticle Physics, \textbf{2025}, 125 (\arxiv{2405.16593}) [\href{https://ui.adsabs.harvard.edu/abs/2025JCAP...01..125R}{26 citations}]

\item[{\color{numcolor}\scriptsize11}] Scholte, Dirk; Saintonge, Am{\'e}lie; Moustakas, John; Catinella, Barbara; Zou, Hu; \etal\ (incl.\ \textbf{A. Anand}), 2024, \doi{10.1093/mnras/stae2477}{The atomic gas sequence and mass-metallicity relation from dwarfs to massive galaxies}, Monthly Notices of the Royal Astronomical Society, \textbf{535}, 2341 (\arxiv{2408.03996}) [\href{https://ui.adsabs.harvard.edu/abs/2024MNRAS.535.2341S}{8 citations}]

\item[{\color{numcolor}\scriptsize10}] Chang, Yu-Ling; Lan, Ting-Wen; Prochaska, J. Xavier; Napolitano, Lucas; \textbf{Anand, Abhijeet}; \etal, 2024, \doi{10.3847/1538-4357/ad6c44}{Probing the Impact of Radio-mode Feedback on the Properties of the Cool Circumgalactic Medium}, The Astrophysical Journal, \textbf{974}, 191 (\arxiv{2405.08314}) [\href{https://ui.adsabs.harvard.edu/abs/2024ApJ...974..191C}{2 citations}]

\item[{\color{numcolor}\scriptsize9}] Galiullin, Ilkham; Rodriguez, Antonio C.; El-Badry, Kareem; Szkody, Paula; \textbf{Anand, Abhijeet}; \etal, 2024, \doi{10.1051/0004-6361/202450734}{Searching for new cataclysmic variables in the Chandra Source Catalog}, Astronomy and Astrophysics, \textbf{690} (\arxiv{2408.00078}) [\href{https://ui.adsabs.harvard.edu/abs/2024A&A...690A.374G}{4 citations}]

\item[{\color{numcolor}\scriptsize8}] \textbf{Anand, Abhijeet}; Guy, Julien; Bailey, Stephen; Moustakas, John; Aguilar, J.; \etal, 2024, \doi{10.3847/1538-3881/ad60c2}{Archetype-based Redshift Estimation for the Dark Energy Spectroscopic Instrument Survey}, The Astronomical Journal, \textbf{168}, 124 (\arxiv{2405.19288}) [\href{https://ui.adsabs.harvard.edu/abs/2024AJ....168..124A}{20 citations}]

\item[{\color{numcolor}\scriptsize7}] DESI Collaboration; Adame, A. G.; Aguilar, J.; Ahlen, S.; Alam, S.; \etal\ (incl.\ \textbf{A. Anand}), 2024, \doi{10.3847/1538-3881/ad3217}{The Early Data Release of the Dark Energy Spectroscopic Instrument}, The Astronomical Journal, \textbf{168}, 58 (\arxiv{2306.06308}) [\href{https://ui.adsabs.harvard.edu/abs/2024AJ....168...58D}{359 citations}]

\item[{\color{numcolor}\scriptsize6}] DESI Collaboration; Adame, A. G.; Aguilar, J.; Ahlen, S.; Alam, S.; \etal\ (incl.\ \textbf{A. Anand}), 2024, \doi{10.3847/1538-3881/ad0b08}{Validation of the Scientific Program for the Dark Energy Spectroscopic Instrument}, The Astronomical Journal, \textbf{167}, 62 (\arxiv{2306.06307}) [\href{https://ui.adsabs.harvard.edu/abs/2024AJ....167...62D}{201 citations}]

\item[{\color{numcolor}\scriptsize5}] Napolitano, Lucas; Pandey, Agnesh; Myers, Adam D.; Lan, Ting-Wen; \textbf{Anand, Abhijeet}; \etal, 2023, \doi{10.3847/1538-3881/ace62c}{Detecting and Characterizing Mg II Absorption in DESI Survey Validation Quasar Spectra}, The Astronomical Journal, \textbf{166}, 99 (\arxiv{2305.20016}) [\href{https://ui.adsabs.harvard.edu/abs/2023AJ....166...99N}{19 citations}]

\item[{\color{numcolor}\scriptsize4}] Ayromlou, Mohammadreza; Kauffmann, Guinevere; \textbf{Anand, Abhijeet}; \& White, Simon D. M., 2023, \doi{10.1093/mnras/stac3637}{The physical origin of galactic conformity: from theory to observation}, Monthly Notices of the Royal Astronomical Society, \textbf{519}, 1913 (\arxiv{2207.02218}) [\href{https://ui.adsabs.harvard.edu/abs/2023MNRAS.519.1913A}{24 citations}]

\item[{\color{numcolor}\scriptsize3}] \textbf{Anand, Abhijeet}; Kauffmann, Guinevere; \& Nelson, Dylan, 2022, \doi{10.1093/mnras/stac928}{Cool circumgalactic gas in galaxy clusters: connecting the DESI legacy imaging survey and SDSS DR16 Mg II absorbers}, Monthly Notices of the Royal Astronomical Society, \textbf{513}, 3210 (\arxiv{2201.07811}) [\href{https://ui.adsabs.harvard.edu/abs/2022MNRAS.513.3210A}{26 citations}]

\item[{\color{numcolor}\scriptsize2}] \textbf{Anand, Abhijeet}; Nelson, Dylan; \& Kauffmann, Guinevere, 2021, \doi{10.1093/mnras/stab871}{Characterizing the abundance, properties, and kinematics of the cool circumgalactic medium of galaxies in absorption with SDSS DR16}, Monthly Notices of the Royal Astronomical Society, \textbf{504}, 65 (\arxiv{2103.15842}) [\href{https://ui.adsabs.harvard.edu/abs/2021MNRAS.504...65A}{44 citations}]

\item[{\color{numcolor}\scriptsize1}] \textbf{Anand, Abhijeet}; Roy, Nirupam; \& Gopal-Krishna, 2019, \doi{10.1088/1674-4527/19/6/83}{Search for H I emission from superdisk candidates associated with radio galaxies}, Research in Astronomy and Astrophysics, \textbf{19}, 83 (\arxiv{1812.06875}) [\href{https://ui.adsabs.harvard.edu/abs/2019RAA....19...83A}{2 citations}]
  \end{list}

  \subsubsection{Preprints \& white papers}
  \begin{list}{}{\cvlist}
    \item[{\color{numcolor}\scriptsize8}] Medina, Gustavo E.; Li, Ting S.; Eadie, Gwendolyn M.; Riley, Alexander H.; Valluri, Monica; \etal\ (incl.\ \textbf{A. Anand}), 2025, \doi{10.48550/arXiv.2508.19351}{The mass of the Milky Way from outer halo stars measured by DESI DR1}, ArXiv, arXiv:2508.19351 (\arxiv{2508.19351})

\item[{\color{numcolor}\scriptsize7}] Rashkovetskyi, M.; Eisenstein, D. J.; Aguilar, J.; Ahlen, S.; \textbf{Anand, Abhijeet}; \etal, 2025, \doi{10.48550/arXiv.2508.20904}{Clustering of DESI galaxies split by thermal Sunyaev-Zeldovich effect}, ArXiv, arXiv:2508.20904 (\arxiv{2508.20904})

\item[{\color{numcolor}\scriptsize6}] Herrera-Alcantar, Hiram K.; Armengaud, Eric; Y{\`e}che, Christophe; Gordon, Calum; Casas, Laura; \etal\ (incl.\ \textbf{A. Anand}), 2025, \doi{10.48550/arXiv.2507.21852}{The Lyman-${\ensuremath{\alpha}}$ Forest from LBGs: First 3D Correlation Measurement with DESI and Prospects for Cosmology}, ArXiv, arXiv:2507.21852 (\arxiv{2507.21852}) [\href{https://ui.adsabs.harvard.edu/abs/2025arXiv250721852H}{1 citations}]

\item[{\color{numcolor}\scriptsize5}] DESI Collaboration; Abdul-Karim, M.; Aguilar, J.; Ahlen, S.; Alam, S.; \etal\ (incl.\ \textbf{A. Anand}), 2025, \doi{10.48550/arXiv.2503.14738}{DESI DR2 Results II: Measurements of Baryon Acoustic Oscillations and Cosmological Constraints}, ArXiv, arXiv:2503.14738 (\arxiv{2503.14738}) [\href{https://ui.adsabs.harvard.edu/abs/2025arXiv250314738D}{386 citations}]

\item[{\color{numcolor}\scriptsize4}] DESI Collaboration; Abdul-Karim, M.; Adame, A. G.; Aguado, D.; Aguilar, J.; \etal\ (incl.\ \textbf{A. Anand}), 2025, \doi{10.48550/arXiv.2503.14745}{Data Release 1 of the Dark Energy Spectroscopic Instrument}, ArXiv, arXiv:2503.14745 (\arxiv{2503.14745}) [\href{https://ui.adsabs.harvard.edu/abs/2025arXiv250314745D}{116 citations}]

\item[{\color{numcolor}\scriptsize3}] DESI Collaboration; Abdul-Karim, M.; Aguilar, J.; Ahlen, S.; Allende Prieto, C.; \etal\ (incl.\ \textbf{A. Anand}), 2025, \doi{10.48550/arXiv.2503.14739}{DESI DR2 Results I: Baryon Acoustic Oscillations from the Lyman Alpha Forest}, ArXiv, arXiv:2503.14739 (\arxiv{2503.14739}) [\href{https://ui.adsabs.harvard.edu/abs/2025arXiv250314739D}{78 citations}]

\item[{\color{numcolor}\scriptsize2}] Brodzeller, A.; Wolfson, M.; Santos, D. M.; Ho, M.; Tan, T.; \etal\ (incl.\ \textbf{A. Anand}), 2025, \doi{10.48550/arXiv.2503.14740}{Construction of the Damped Ly${\ensuremath{\alpha}}$ Absorber Catalog for DESI DR2 Ly${\ensuremath{\alpha}}$ BAO}, ArXiv, arXiv:2503.14740 (\arxiv{2503.14740}) [\href{https://ui.adsabs.harvard.edu/abs/2025arXiv250314740B}{11 citations}]

\item[{\color{numcolor}\scriptsize1}] Han, Jiwon Jesse; Dey, Arjun; Price-Whelan, Adrian M.; Najita, Joan; Schlafly, Edward F.; \etal\ (incl.\ \textbf{A. Anand}), 2023, \doi{10.48550/arXiv.2306.11784}{NANCY: Next-generation All-sky Near-infrared Community surveY}, ArXiv, arXiv:2306.11784 (\arxiv{2306.11784}) [\href{https://ui.adsabs.harvard.edu/abs/2023arXiv230611784H}{6 citations}]
  \end{list}
\fi

\subsection{Selected visits, talks and conferences}
$^i$\emph{Invited Talks}, $^c$\emph{Contributed Talks}, $^s$\emph{Schools \& Workshops}
\begin{list}{}{\cvlist}
\item $^i$ Feb 2024:  \emph{Extragalactic Seminar}, MPA Garching, Germany, Feb 29, 2024 (\textit{Talk}).
\item $^i$ Feb 2024:  \emph{Cosmology Seminar}, MPA Garching, Germany, Feb 27, 2024 (\textit{Talk}).
    %\item $^i$ Aug 2023:  \emph{DESI Data Systems and Management Meeting}, Oct 31, 2023 (\textit{Virtual Talk}).
%\item $^i$ Aug 2023:  \emph{DESI Data Systems and Management Meeting}, Aug 29, 2023 (\textit{Virtual Talk}). 
\item $^c$ Dec 2023:  \emph{DESI Collaboration Winter Meeting}, Kona, Hawaii, Dec 11 - Dec 14, 2023 (\textit{Virtual talk}).
\item $^c$ Jul 2023:  \emph{DESI Collaboration Summer Meeting}, Durham, UK, Jul 17 - Jul 21, 2023 (\textit{Talk}). 
\item $^c$ Mar 2023:  \emph{DESI Absorber Science Workshop}, Mar 28, 2023 (\textit{ Virtual Talk}).   
\item $^c$ Dec 2022:  \emph{DESI Collaboration Winter Meeting}, Cancun, Mexico, 5-9 Dec, 2022 (\textit{Talk}).   
%\item $^c$ Sep 2022:  \emph{\href{https://sites.google.com/unimib.it/gas2022/home}{What matter(s) around galaxies 2022}}, Champoluc, Italy, 12-16 Sep, 2022 (\textit{Virtual}).   
\item $^c$ May 2022:  \emph{\href{https://indico.ph.tum.de/event/7018/}{Gas Flows around Galaxies}}, MPA, Garching, 24 May, 2022 (\textit{Talk}).   
\item $^c$ Apr 2022:  \emph{\href{https://www.stsci.edu/contents/events/stsci/2022/april/galaxy-clusters-2022-challenging-our-cosmological-perspectives}{Galaxy Clusters 2022}}, STScI 25-29 Apr, 2022 (\textit{e-Poster}).   
\item $^i$ Mar 2022:  \emph{\href{https://zah.uni-heidelberg.de/research-groups\#c2659}{Computational Galaxy Formation and Evolution Group}}, ZAH/ITA, University of Heidelberg, 14-16 Mar, 2022 (\textit{Visitor}).   
\item $^i$ Feb 2022:  \emph{High Energy Group Seminar}, MPE Garching, 15 Feb, 2022 (\textit{Virtual}).   
\item $^i$ Feb 2022:  \emph{DESI Group Seminar}, Berkeley Lab, 1 Feb, 2022 (\textit{Virtual}).   
\item $^i$ Dec 2021:  \emph{\href{https://pweb.cfa.harvard.edu/calendar/event/9298}{Galaxies \& Cosmology Seminar}}, CfA Harvard, 13 Dec, 2021 (\textit{Virtual}).   
\item $^i$ Sep 2021:  \textit{STARs lab}, Arizona State University, 24 Sep, 2021 (\textit{Virtual}).   

\item $^c$ Nov 2021:  \emph{\href{https://www.bilibili.com/video/BV1nv411M7w3}{e-Poster \& talk}}, \emph{\href{https://kiaa.pku.edu.cn/KooGig_junior21/Home.htm}{KooGiG-Junior'21}}, KIAA, Peking University, 1-5 Nov, 2021 (\textit{Virtual}).   

\item $^c$ Sep 2021:  \emph{\href{https://ag2021.astronomische-gesellschaft.de/view_splinter.php?session=Stars}{German Astronomical Society Meeting 2021}}, 16 Sep, 2021 (\textit{Virtual})   
\item $^c$ Aug 2021:  \emph{\href{https://jhu2021.sdss.org/}{SDSS-IV Collaboration Meeting 2021}}, 12 Aug, 2021 (\textit{Virtual + e-poster})   
\item $^s$ Jul 2021:  \emph{MIAPP Workshop: High-Energy Plasma Physics Phenomena in Astrophysics}, MIAPP, Garching, Jul 19 - 30, 2021 (\textit{Virtual participant}).  

\item $^c$ Jun 2021:  \emph{MPA Galaxy Group Retreat}, Fraueninsel, Chiemsee, 14-17 June, 2021 (\textit{Talk}).  
\item $^s$ Jan 2021:  \emph{Fundamental of Gaseous Halos}, KITP, UCSB, 11 Jan - 5 Mar, 2021 (\textit{Virtual participant}).  
\item $^c$ Oct 2020:  \emph{Galaxy Introductory Symposium}, MPA, Garching, 27 Oct, 2020 (\textit{Virtual}).  
\item $^c$ Oct 2020:  \emph{Institute Seminar}, MPA, Garching, 26 Oct, 2020 (\textit{Virtual}).  

\item $^c$ Oct 2019:  \emph{\href{https://wwwmpa.mpa-garching.mpg.de/conf/berlincgm2019/}{CGM 2019: CGM Conference}}, Berlin, 3 Oct - 5 Oct 2019 (\textit{Participation}).  
\item $^s$ Aug 2019:  \emph{Summer School: Galaxy Formation}, AKSS, Spetses, 28 Aug - 5 Sept 2019.  
\item $^s$ Nov 2018:  \emph{Python for HPC}, attended at MPCDF, Garching, 20-21 Nov 2018.  

\end{list}

\subsection{Mentoring}
\begin{list}{}{\cvlist}
\item Joanne Tan (MPA Garching PhD) (CGM absorbers in TNG-100), 2024-present
\item Corey Dodeson (UC Berkeley UG) (short-term reading project on CGM), 2023
\item Dylan Green (UC Irvine PhD) (as part of DESI mentorship program), 2023
\end{list}

\subsection{Honors \& Awards}
\begin{list}{}{\cvlist}
  \item DESI Early Career Travel Grant (USD 4000).
  \item DESI Postdoc Fellow, 2022-2025.
  \item IMPRS PhD Fellowship, 2018--2022.
  \item UGC - Junior Research Fellowship, 2017-2018 (INR 390,000).
  \item DST\footnote{Department of Science \& Technology, Govt. of India} - Higher Studies Scholarship, 2012-2017.

\end{list}

\subsection{Computer skills}
\begin{list}{}{\cvlist}
\item Language/Packages ---  
    python, numpy, matplotlib, scipy, astropy
\item OS/Tools --- Linux, Mac, \LaTeX, bash scripting, jupyter notebooks, slurm manager, git
\end{list}
% \ifdefined\withpubs
%     \newpage
% \fi

\subsection{Professional service \& activities}
\begin{list}{}{\cvlist}
    \item Active Referee ---
    \href{https://en.wikipedia.org/wiki/Astronomy_%26_Astrophysics}{A\&A}
    , \href{https://en.wikipedia.org/wiki/The_Astrophysical_Journal}{ApJ}, Internal Reviewer for DESI collaboration papers
    \item Collaboration ---
        \href{https://www.desi.lbl.gov/}{Dark Energy Survey Instrument (DESI)}, NANCY
    \item Member \& contributor (github repo) ---
    \begin{itemize}
      \item[] \href{https://github.com/desihub/redrock}{redrock}: Redshift fitting for spectroperfectionism.
      \item[] \href{https://github.com/desihub}{desihub}: Public code associated with DESI.
    \end{itemize}
    \item Organizer ---
        \href{https://inpa.lbl.gov/events/}{INPA Weekly Seminar}, Physics Division, LBNL
    %\item Mentoring ---
    %    Corey Dodeson, UC Berkeley UG (short-term spring project), Dylan Green, UC Irvine (as part of DESI mentorship program)
    \item My papers (\href{https://www.mpa-garching.mpg.de/964620/hl202107}{I}, \href{https://www.mpa-garching.mpg.de/1066558/hl202211?c=27981}{II}) featured in monthly research highlight of MPA and appeared on \href{https://astrobites.org/2021/05/06/cool-metal-gas-search-thanks-it-was-automated/}{astrobites}.
    \item Gave \href{https://theinterviewportal.com/2020/03/13/astrophysicist-interview-8/}{interview} to \href{https://theinterviewportal.com/}{The Interview Portal} to help Indian college students plan their careers. 
    \item Member of Local organizing Committee of VII\textsuperscript{th} IMPRS Student Symposium, 4-5 Apr, 2019.
  \item Served as online physics tutor at E-acharya (more than 20 students)\footnote{an initiative to help poor students in sub-urban and rural areas of Bihar, India.} from June 2017 - Aug 2018.
    \item Advanced Physics tutor at Chegg India from Jul 2016 - Dec 2017. Solving advance physics problems and clearing doubts of students from all over the world.
  \item Organized and coordinated scientific and technical workshops ($\sim$ 150 participants) in \emph{Pravega}\footnote{Annual Science Festival of IISc}, in Jan'14.
    \item Conducted experiments and demonstrations for open day celebration at IISc during my undergraduate.
\end{list}

\end{document}
